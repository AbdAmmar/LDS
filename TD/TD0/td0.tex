\documentclass[11pt,a4paper]{article}

\usepackage[T1]{fontenc}
\usepackage[french]{babel}
\usepackage{geometry}
\usepackage{hyperref}

\usepackage{../tdstyle}

\geometry{margin=2.5cm}

\title{TD0 -- Installation de Linux avec WSL}
\author{Abdallah Ammar}
\date{\today}

\begin{document}
\maketitle


\section*{Objectifs du TD}

Ce premier TD a pour objectif de préparer l'environnement de travail pour le module
\textbf{Langages de script}.

Les outils et scripts utilisés dans ce cours s'exécuteront dans un environnement
\textbf{Linux / Unix}.

\begin{itemize}
    \item \textbf{Étudiants sous Windows} : ce TD vous guide pas à pas dans
    l'installation d'un environnement Linux grâce à \textbf{WSL (Windows Subsystem for Linux)}.
    
    \item \textbf{Étudiants disposant déjà de Linux} : aucune installation n'est requise.
    Vous pouvez passer directement aux prochains TD.
    
    \item \textbf{Étudiants sous macOS} : aucun système supplémentaire n'est nécessaire.
    Le terminal est disponible par défaut et le shell \texttt{bash} est déjà installé.
    Depuis les versions récentes de macOS, le shell utilisé par défaut est \texttt{zsh}.
    Pour connaître le shell actuellement utilisé, tapez la commande :
\begin{terminal}
\prompt\ \shcmd{echo} \$SHELL
\end{terminal}

\begin{minted}{bash}
#!/bin/bash
echo "Bonjour"
ls -l
\end{minted}
    Si nécessaire, il est possible de lancer \texttt{bash} simplement en tapant :
    \begin{minted}{bash}
        bash
    \end{minted}
    dans le terminal.
\end{itemize}


\section*{Qu'est-ce que WSL ?}

WSL est une fonctionnalité de Windows qui permet d'exécuter un système Linux
directement sous Windows, sans machine virtuelle lourde ni redémarrage.

Avantages :
\begin{itemize}
    \item Pas besoin de dual-boot
    \item Intégration avec Windows
    \item Accès aux commandes Linux standards
\end{itemize}

\section*{Prérequis}

\begin{itemize}
    \item Windows 10 (version 2004 ou plus récente) ou Windows 11
    \item Connexion Internet
    \item Droits administrateur sur la machine
\end{itemize}

\section*{Étape 1 : Ouvrir un terminal Windows}

\begin{enumerate}
    \item Appuyez sur \texttt{Win + R}
    \item Tapez \texttt{cmd}
    \item Appuyez sur \texttt{Ctrl + Shift + Entrée} pour ouvrir en mode administrateur
\end{enumerate}

\section*{Étape 2 : Installer WSL}

Dans le terminal, tapez la commande suivante :

%\begin{lstlisting}
%wsl --install
%\end{lstlisting}

Cette commande :
\begin{itemize}
    \item Active WSL
    \item Installe le noyau Linux
    \item Installe par défaut Ubuntu
\end{itemize}

\textbf{Redémarrez votre ordinateur} lorsque cela vous est demandé.

\section*{Étape 3 : Premier lancement d'Ubuntu}

Après le redémarrage :
\begin{itemize}
    \item Lancez \textbf{Ubuntu} depuis le menu Démarrer
    \item Patientez pendant l'installation
    \item Choisissez un nom d'utilisateur et un mot de passe Linux
\end{itemize}

\section*{Étape 4 : Vérification de l'installation}

Dans le terminal Ubuntu, tapez :

\begin{minted}{bash}
lsb_release -a
\end{minted}

Puis :

\begin{minted}{bash}
uname -a
\end{minted}

Si ces commandes s'exécutent sans erreur, l'installation est réussie.

\section*{Problèmes fréquents}

\begin{itemize}
    \item \textbf{La commande \texttt{wsl} n'est pas reconnue} : Windows trop ancien
    \item \textbf{Erreur de virtualisation} : activer la virtualisation dans le BIOS
\end{itemize}

\section*{Ressources}

Documentation officielle :
\begin{itemize}
    \item \url{https://learn.microsoft.com/fr-fr/windows/wsl/}
\end{itemize}

\end{document}
