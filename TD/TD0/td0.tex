\documentclass[11pt,a4paper]{article}

\usepackage[T1]{fontenc}
\usepackage[french]{babel}
\usepackage{geometry}

\usepackage{../../ldsstyle}

\geometry{margin=2.5cm}

\title{TD0 -- Langages de Script}
\author{Abdallah Ammar}
\date{\today}

\begin{document}
\maketitle


Ce TD a pour objectif de préparer l'environnement de travail pour le module
\textbf{Langages de script}.
Les outils et scripts utilisés dans ce cours s'exécuteront dans un environnement
\textbf{Linux / Unix}.

\begin{warningbox}
Le symbole \texttt{\$} affiché au début des commandes correspond à l'invite du terminal.
Il sert uniquement à indiquer que la commande doit être tapée, mais ne fait pas partie
de la commande elle-même.
\end{warningbox}

\begin{itemize}
    \item \textbf{Étudiants sous Windows} : ce TD vous guide pas à pas dans
    l'installation d'un environnement Linux grâce à \textbf{WSL (Windows Subsystem for Linux)}.
    
    \item \textbf{Étudiants disposant déjà de Linux} : aucune installation n'est requise.
    Vous pouvez passer directement aux prochains TD.
    
    \item \textbf{Étudiants sous macOS} : aucun système supplémentaire n'est nécessaire.
    Le terminal est disponible par défaut et le shell \texttt{bash} est déjà installé.
    Depuis les versions récentes de macOS, le shell utilisé par défaut est \texttt{zsh}.
    Pour connaître le shell actuellement utilisé, tapez la commande :
\begin{terminal}
\prompt\ \shcmd{echo} \$SHELL
\end{terminal}

    Si nécessaire, il est possible de lancer \texttt{bash} simplement en tapant :
    \begin{terminal}
    \prompt\ \shcmd{bash}
    
    \prompt\ \tcomment{ou}
    
    \prompt\ \shcmd{/bin/bash}
    \end{terminal}
    dans le terminal.
    Pour confirmer que \texttt{bash} est actif, utilisez cette commande :
    \begin{terminal}
    \prompt\ \shcmd{echo} \$0
    \end{terminal}
    Optionnellement, pour définir \texttt{bash} comme environnement par défaut, utilisez 
    la commande :
    \begin{terminal}
    \prompt\ \shcmd{chsh} -s /bin/bash
    \end{terminal}
\end{itemize}

    Pour installer des programmes supplémentaires en ligne de commande,
    il est recommandé d'utiliser le gestionnaire de paquets \texttt{Homebrew}.
    L'installation de Homebrew se fait une seule fois avec les commandes suivantes :
    \begin{terminal}
    \prompt\ brew\_link="https://raw.githubusercontent.com/Homebrew/install/HEAD/install.sh"

    \prompt\ \shcmd{curl} -fsSL \$brew\_link -o install\_brew.sh
    
    \prompt\ \shcmd{bash} install\_brew.sh
    \end{terminal}

    Après l'installation, si la commande \texttt{brew} n'est pas reconnue,
    \begin{terminal}
    \prompt\ \shcmd{brew} {-}{-}version
    \end{terminal}
    il est nécessaire d'ajouter Homebrew au \texttt{PATH}.
    \begin{terminal}
    \prompt\ \shcmd{echo} 'eval "\$(/opt/homebrew/bin/brew shellenv)"' \verb|>>| \textasciitilde/.bashrc
    
    \prompt\ \shcmd{eval} "\$(/opt/homebrew/bin/brew shellenv)"
    \end{terminal}
    

\section*{Qu'est-ce que WSL ?}

\href{https://learn.microsoft.com/fr-fr/windows/wsl/}{WSL}
est une fonctionnalité de Windows qui permet d'exécuter un système Linux
directement sous Windows, sans machine virtuelle lourde ni redémarrage.

Avantages :
\begin{itemize}
    \item Pas besoin de dual-boot
    \item Intégration avec Windows
    \item Accès aux commandes Linux standards
\end{itemize}

\section*{Prérequis}

\begin{itemize}
    \item Windows 10 (version 2004 ou plus récente) ou Windows 11
    \item Connexion Internet
    \item Droits administrateur sur la machine
\end{itemize}

\section*{Étape 1 : Ouvrir un terminal Windows}

\begin{enumerate}
    \item Appuyez sur \texttt{Win + R}
    \item Tapez \texttt{cmd}
    \item Appuyez sur \texttt{Ctrl + Shift + Entrée} pour ouvrir en mode administrateur
\end{enumerate}

\section*{Étape 2 : Installer WSL}

Dans le terminal, tapez la commande suivante :

\begin{terminal}
\prompt\ \shcmd{wsl} {-}{-}install
\end{terminal}

Cette commande :
\begin{itemize}
    \item Active WSL
    \item Installe le noyau Linux
    \item Installe par défaut Ubuntu
\end{itemize}

\textbf{Redémarrez votre ordinateur} lorsque cela vous est demandé.

\section*{Étape 3 : Premier lancement d'Ubuntu}

Après le redémarrage :
\begin{itemize}
    \item Lancez \textbf{Ubuntu} depuis le menu Démarrer
    \item Patientez pendant l'installation
    \item Choisissez un nom d'utilisateur et un mot de passe Linux
\end{itemize}

\section*{Étape 4 : Vérification de l'installation}

Dans le terminal Ubuntu, tapez :

\begin{terminal}
\prompt\ \shcmd{lsb\_release} -a
\end{terminal}

Puis :

\begin{terminal}
\prompt\ \shcmd{uname} -a
\end{terminal}

Si ces commandes s'exécutent sans erreur, l'installation est réussie.

\section*{Problèmes fréquents}

\begin{itemize}
    \item \textbf{Erreur 0x80370102} : la virtualisation du CPU n'est pas activée.
    Il faut activer la virtualisation dans le BIOS/UEFI.
    Depuis Windows, allez dans \textit{Paramètres} $\rightarrow$ \textit{Système} $\rightarrow$
    \textit{Récupération} $\rightarrow$ \textit{Démarrage avancé}.
    Dans le menu de démarrage, accédez aux options UEFI et activez la virtualisation du CPU
    (les menus peuvent varier selon les machines).
    Voir \href{https://askubuntu.com/questions/1264102/wsl-2-wont-run-ubuntu-error-0x80370102}{cette} discussion par exemple.

    \item \textbf{Erreur 0x800701bc} : WSL est en version 1.
    Mettez à jour votre système Windows et assurez-vous d'utiliser WSL~2.
    Voir \href{https://learn.microsoft.com/windows/wsl/install}{ici}.

    \item \textbf{Erreur 0xc03a001a} : la distribution Ubuntu est installée dans un dossier compressé.
    Il faut désactiver la compression du dossier \\
    \texttt{C:\textbackslash Users\textbackslash <nom>\textbackslash AppData\textbackslash Local\textbackslash Packages\textbackslash CanonicalGroupLimited.Ubuntu}
    et de tous ses sous-dossiers.
    Si le dossier \texttt{AppData} n'est pas visible, activez l'affichage des dossiers cachés.
    Voir \href{https://github.com/microsoft/WSL/issues/4299}{ici}.

    \item \textbf{Ubuntu ne se configure pas au premier lancement} :
    la distribution n'est peut-être pas installée.
    Installez manuellement Ubuntu~22.04 LTS depuis le Microsoft Store.

    \item \textbf{Erreur 0x80070050} :
    essayez de mettre à jour Windows ou de réinstaller Ubuntu~22.04.
    Si le problème persiste, un diagnostic de Windows Update peut être nécessaire.

    \item \textbf{Erreur 0x80004002} :
    vérifiez que les fonctionnalités de virtualisation (Hyper-V ou
    \textit{Virtual Machine Platform}) sont activées dans
    \textit{Fonctionnalités Windows}.

    \item \textbf{Erreur \emph{Le chemin d'accès est introuvable}} :
    lors du lancement d'Ubuntu, essayez de mettre à jour WSL en lançant la commande suivante
    dans le terminal Windows (PowerShell) :
\begin{terminal}
\prompt\ \shcmd{wsl} {-}{-}update
\end{terminal}
\end{itemize}

\end{document}
