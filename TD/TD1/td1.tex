\documentclass[11pt,a4paper]{article}

\usepackage[T1]{fontenc}
\usepackage[french]{babel}
\usepackage{geometry}

\usepackage{../tdstyle}

\geometry{margin=2.5cm}

\title{TD1 -- Langages de Script}
\author{Abdallah Ammar}
\date{\today}

\begin{document}
\maketitle



\begin{warningbox}
Le symbole \texttt{\$} affiché au début des commandes correspond à l'invite du terminal.
Il sert uniquement à indiquer que la commande doit être tapée, mais ne fait pas partie
de la commande elle-même.
\end{warningbox}


\begin{warningbox}
Il est fortement recommandé de taper les commandes à la main afin de se familiariser
avec le terminal.
\end{warningbox}


Dans le terminal, il est possible de réutiliser et de modifier des commandes déjà tapées.
Les flèches du clavier permettent de naviguer dans l'historique des commandes :
\begin{itemize}
    \item flèche vers le haut : commande précédente ;
    \item flèche vers le bas : commande suivante.
\end{itemize}

Il est également possible d'utiliser la touche \texttt{TAB} pour compléter automatiquement
les noms de commandes, de fichiers ou de répertoires.
Cette fonctionnalité permet d'éviter les fautes de frappe et d'accélérer la saisie.

L'historique des commandes peut être affiché avec la commande suivante :
\begin{terminal}
\prompt\ \shcmd{history}
\end{terminal}

Les lignes commençant par le symbole \texttt{\#} sont des commentaires.
Elles servent à expliquer ou documenter des commandes, mais ne sont pas exécutées
par le shell.

\bigskip


% --- --- --- --- --- --- --- --- --- --- --- --- --- --- --- --- --- --- --- --- --- ---

\section{Premiers pas}

\rsubsection{}
Certaines commandes sont intégrées directement dans le shell
(commandes internes ou \textit{built-in}),
tandis que d'autres sont des programmes externes.

La commande \texttt{type} permet de savoir de quel type est une commande.
Par exemple :

\begin{terminal}
\prompt\ \shcmd{type} \shcmd{cd}
\end{terminal}

À l'aide de cette commande, déterminez si les commandes suivantes sont
des commandes internes ou des commandes externes :

\begin{itemize}
    \item \texttt{cd}
    \item \texttt{echo}
    \item \texttt{ls}
    \item \texttt{pwd}
    \item \texttt{type}
    \item \texttt{mkdir}
    \item \texttt{useradd}
    \item \texttt{date}
    \item \texttt{time}
    \item \texttt{sleep}
    \item \texttt{touch}
    \item \texttt{cat}
    \item \texttt{grep}
\end{itemize}

% ---

\rsubsection{}
Savoir se documenter est une compétence essentielle sous Linux.
La documentation permet de comprendre le fonctionnement d'une commande
et de découvrir ses options.

La commande \texttt{man} fournit une page de manuel concise et standard pour les
commandes externes, tandis que la commande \texttt{info} propose une documentation
plus détaillée, organisée comme un ensemble de chapitres.
La commande \texttt{help} affiche une aide intégrée au shell et s'applique
principalement aux commandes internes.

\begin{enumerate}
    \item Consultez la documentation de la commande \texttt{ls}.
    \item Trouvez l'option permettant d'afficher les fichiers cachés, puis utilisez-la
          pour identifier les fichiers cachés présents dans votre répertoire personnel.
    \item Expliquez pourquoi la commande \texttt{man cd} ne fonctionne pas. Consultez 
          l'aide de la commande \texttt{cd}.
    \item Consultez la documentation de \texttt{ls} avec \texttt{info}.
\end{enumerate}

\subsection*{Commandes utiles}

\begin{terminal}
\prompt\ \shcmd{man} \shcmd{ls}

\prompt\ \shcmd{ls} -a \$HOME

\prompt\ \shcmd{help} \shcmd{cd}

\prompt\ \shcmd{info} \shcmd{ls}
\end{terminal}

% ---

\rsubsection{}
Pour installer un nouveau programme sous Linux, il est recommandé de commencer
par mettre à jour la liste des paquets disponibles avec \texttt{apt update}.
Cela permet au système de connaître les versions les plus récentes des logiciels.
\begin{terminal}
\prompt\ \shcmd{sudo apt update}
\end{terminal}

La commande \texttt{sudo} permet d'exécuter une commande avec les droits
d'administrateur.
Elle est nécessaire pour installer des programmes ou modifier le système.
Lors de son utilisation, un mot de passe peut être demandé.

\begin{warningbox}
Utilisez \texttt{sudo} uniquement lorsque cela est nécessaire et
vérifiez toujours la commande avant de l'exécuter.
\end{warningbox}

Ensuite, on installe le programme souhaité à l'aide de la commande
\texttt{apt install}.

Par exemple, installez puis lancez le programme \texttt{sl} :
\begin{terminal}
\prompt\ \shcmd{sudo apt install sl}

\prompt\ \shcmd{sl}
\end{terminal}

Installez maintenant le programme \texttt{cowsay},
qui affiche un message sous forme de dessin ASCII :
\begin{terminal}
\prompt\ \shcmd{sudo apt install cowsay}
\end{terminal}

Puis lancez-le avec un message de votre choix :
\begin{terminal}
\prompt\ \shcmd{cowsay} "Bonjour Linux"
\end{terminal}

% ---

\rsubsection{}
Dans cet exercice, vous allez créer une petite arborescence afin de comprendre
l'effet des chemins relatifs et absolus.

Après chaque changement de répertoire avec la commande \texttt{cd},
affichez systématiquement le répertoire courant à l'aide de la commande \texttt{pwd}.

\begin{enumerate}
    \item Allez dans votre répertoire personnel.
    \item Créez un répertoire nommé \texttt{TD2}.
    \item Entrez dans \texttt{TD2}.
    \item Créez un répertoire \texttt{TD2-bis}.
    \item Entrez dans \texttt{TD2-bis}.
    \item Créez les répertoires suivants :
    \begin{itemize}
        \item \texttt{TD2-bis-1}
        \item \texttt{TD2-bis-2}
        \item \texttt{TD2-bis-3}
    \end{itemize}
    \item Entrez dans \texttt{TD2-bis-3}.
    \item Depuis \texttt{TD2-bis-3}, créez un répertoire nommé \texttt{test}
    à l'intérieur de \texttt{TD2-bis-2}, sans vous déplacer dans ce dernier.
    \item Revenez au répertoire \texttt{TD2} avec une seule commande.
    \item Revenez ensuite dans votre répertoire personnel.
\end{enumerate}

\subsection*{Affichage de l'arborescence}

Affichez l'arborescence complète du répertoire \texttt{TD2}.

\begin{terminal}
\prompt\ \shcmd{tree} TD2
\end{terminal}

Si la commande \texttt{tree} n'est pas disponible, installez-la :
\begin{terminal}
\prompt\ \shcmd{sudo apt update}

\prompt\ \shcmd{sudo apt install tree}
\end{terminal}

% ---

\rsubsection{}
Dans cet exercice, vous allez créer un fichier texte et y écrire plusieurs lignes.

\begin{enumerate}
    \item Placez-vous dans le répertoire \texttt{TD2}.
    \item Créez un fichier nommé \texttt{notes.txt}.
    \item Écrivez une première ligne dans le fichier.
    \item Ajoutez deux lignes supplémentaires sans effacer les précédentes.
    \item Affichez le contenu du fichier.
\end{enumerate}

\subsection*{Commandes utiles}

\begin{terminal}
\prompt\ \shcmd{touch} notes.txt

\prompt\ \shcmd{echo} "Première ligne" \texttt{>} notes.txt

\prompt\ \shcmd{echo} "Deuxième ligne" \texttt{>>} notes.txt

\prompt\ \shcmd{echo} "Troisième ligne" \texttt{>>} notes.txt

\prompt\ \shcmd{cat} notes.txt
\end{terminal}

% --- --- --- --- --- --- --- --- --- --- --- --- --- --- --- --- --- --- --- --- --- ---

\section{Informations système}

Linux permet d'obtenir rapidement des informations sur la machine
(processeur, mémoire, etc.) ainsi que sur le système.

\rsubsection{} 
Complétez la fiche suivante à l'aide de commandes Linux :
\begin{itemize}
    \item Nom de la machine :
    \item Nom de l'utilisateur courant :
    \item Système d'exploitation (distribution) :
    \item Version du noyau Linux :
    \item Architecture du processeur (ex. \texttt{x86\_64}, \texttt{aarch64}) :
\end{itemize}

\subsection*{Commandes utiles}

\begin{terminal}
\prompt\ \shcmd{hostname}

\prompt\ \shcmd{whoami}

\prompt\ \shcmd{uname} \texttt{-a}

\prompt\ \shcmd{uname} \texttt{-m}

\prompt\ \shcmd{lsb\_release} \texttt{-a}
\end{terminal}

% ---

\rsubsection{}
Le processeur (CPU) est l'un des éléments centraux d'une machine.
\begin{enumerate}
    \item Combien de processeurs logiques possède votre machine ?
    \item Quel est le modèle du processeur ?
    \item Comparez vos résultats avec un autre étudiant.
\end{enumerate}

\subsection*{Commandes utiles}

\begin{terminal}
\prompt\ \shcmd{lscpu}

\prompt\ \shcmd{nproc}
\end{terminal}

% ---

\rsubsection{}
La mémoire vive est utilisée pour exécuter les programmes en cours.

\begin{enumerate}
    \item Quelle est la quantité totale de mémoire vive ?
    \item Quelle quantité est actuellement utilisée ?
\end{enumerate}

\subsection*{Commandes utiles}

\begin{terminal}
\prompt\ \shcmd{free} \texttt{-h}
\end{terminal}

% ---

\rsubsection{} 
Le programme \texttt{neofetch} permet d'afficher un résumé visuel de informations système.
Installez-le à l'aide des commandes suivantes :
\begin{terminal}
\prompt\ \shcmd{sudo apt update}

\prompt\ \shcmd{sudo apt install} neofetch
\end{terminal}

Puis lancez-le :
\begin{terminal}
\prompt\ \shcmd{neofetch}
\end{terminal}
et comparez les informations affichées avec celles que vous avez obtenues
à l'aide des commandes précédentes.

% ---

\rsubsection{}
Linux organise les disques de stockage à l'aide de points de montage.

\begin{enumerate}
    \item Quelle est la taille de l'espace disque contenant votre répertoire personnel ?
    \item Combien d'espace est actuellement disponible ?
    \item Quelle est la taille de votre répertoire personnel ?
\end{enumerate}

\subsection*{Commandes utiles}

\begin{terminal}
\prompt\ \shcmd{df} \texttt{-h} \texttt{\textasciitilde}

\prompt\ \shcmd{du} \texttt{-h} {-}{-}max-depth=1 \texttt{\textasciitilde}
\end{terminal}

% --- --- --- --- --- --- --- --- --- --- --- --- --- --- --- --- --- --- --- --- --- ---





\end{document}
