\documentclass[11pt,a4paper]{article}

\usepackage[T1]{fontenc}
\usepackage[french]{babel}
\usepackage{geometry}

\usepackage{../tdstyle}

\geometry{margin=2.5cm}

\title{TD1 -- Langages de Script}
\author{Abdallah Ammar}
\date{\today}

\begin{document}
\maketitle



\begin{warningbox}
Il est fortement recommandé de taper les commandes à la main afin de se familiariser
avec le terminal.
\end{warningbox}



\section*{Utilisation du terminal}

Dans le terminal, il est possible de réutiliser et de modifier des commandes déjà tapées.
Les flèches du clavier permettent de naviguer dans l'historique des commandes :
\begin{itemize}
    \item flèche vers le haut : commande précédente,
    \item flèche vers le bas : commande suivante.
\end{itemize}
Il est fortement recommandé d'utiliser cette fonctionnalité afin d'éviter de retaper
entièrement des commandes longues et de corriger plus facilement des erreurs.
L'historique des commandes peut être affiché avec la commande :
\begin{terminal}
\prompt\ \shcmd{history}
\end{terminal}


\bigskip

\section{Carte d'identité de votre machine}

Linux permet d'obtenir rapidement des informations sur la machine et le système.

\subsection*{Travail à faire}

Complétez la fiche suivante à l'aide de commandes Linux :

\begin{itemize}
    \item Nom de la machine :
    \item Nom de l'utilisateur courant :
    \item Système d'exploitation :
    \item Version du noyau Linux :
    \item Architecture du processeur (ex. \texttt{x86\_64}, \texttt{aarch64}) :
\end{itemize}

\subsection*{Commandes utiles}

\begin{terminal}
\prompt\ \shcmd{hostname}

\prompt\ \shcmd{whoami}

\prompt\ \shcmd{uname} \texttt{-a}

\prompt\ \shcmd{uname} \texttt{-m}

\prompt\ \shcmd{lsb\_release} \texttt{-a}
\end{terminal}

\section{Le processeur}

Le processeur (CPU) est l'un des éléments centraux d'une machine.

\subsection*{Travail à faire}

\begin{enumerate}
    \item Combien de processeurs logiques possède votre machine ?
    \item Quel est le modèle du processeur ?
    \item Comparez vos résultats avec un autre étudiant.
\end{enumerate}

\subsection*{Commandes utiles}

\begin{terminal}
\prompt\ \shcmd{lscpu}

\prompt\ \shcmd{nproc}
\end{terminal}

\section{La mémoire vive (RAM)}

La mémoire vive est utilisée pour exécuter les programmes en cours.

\subsection*{Travail à faire}

\begin{enumerate}
    \item Quelle est la quantité totale de mémoire vive ?
    \item Quelle quantité est actuellement utilisée ?
\end{enumerate}

\subsection*{Commandes utiles}

\begin{terminal}
\prompt\ \shcmd{free} \texttt{-h}
\end{terminal}



\section{Le disque et l'espace de stockage}

Linux organise les disques à l'aide de points de montage.

\subsection*{Travail à faire}

\begin{enumerate}
    \item Quelle est la taille de l'espace disque contenant votre répertoire personnel ?
    \item Combien d'espace est actuellement disponible ?
    \item Quelle est la taille de votre répertoire personnel ?
\end{enumerate}

\subsection*{Commandes utiles}

\begin{terminal}
\prompt\ \shcmd{df} \texttt{-h} \texttt{\textasciitilde}

\prompt\ \shcmd{du} \texttt{-h} {-}{-}max-depth=1 \texttt{\textasciitilde}
\end{terminal}




\section*{Pour aller plus loin -- Exercice ludique (optionnel)}

Installez et lancez le programme \texttt{neofetch}.
Observez les informations affichées et identifiez celles que vous reconnaissez
(processseur, mémoire, système).

\begin{terminal}
\prompt\ \shcmd{sudo} apt update

\prompt\ \shcmd{sudo} apt install neofetch

\prompt\ \shcmd{neofetch}
\end{terminal}

\end{document}
