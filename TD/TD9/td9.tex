\documentclass[11pt,a4paper]{article}

\usepackage[T1]{fontenc}
\usepackage[french]{babel}
\usepackage{geometry}

\usepackage{../tdstyle}

\geometry{margin=2.5cm}

\title{TD9 -- Langages de script}
\author{Abdallah Ammar}
\date{\today}

\begin{document}
\maketitle

% --------------------------------------------------
\section*{Objectifs du TD}

Ce TD a pour objectif de comprendre comment créer une commande personnalisée
sous Linux, à la manière de la commande \texttt{fortune}.
À l'issue de la séance, vous devrez être capables de :
\begin{itemize}
    \item écrire un script bash simple ;
    \item rendre un script exécutable ;
    \item placer un script dans le \texttt{PATH} ;
    \item créer une commande réutilisable ;
    \item combiner votre commande avec \texttt{cowsay}.
\end{itemize}

\bigskip

\begin{warningbox}
Ce TD est ludique et créatif.
Aucune évaluation ne portera sur le contenu des citations,
mais sur le bon fonctionnement de la commande.
\end{warningbox}

% --------------------------------------------------
\section{Installation des outils}

Installez les programmes nécessaires :

\begin{terminal}
\prompt\ \shcmd{sudo} apt update
\prompt\ \shcmd{sudo} apt install cowsay
\end{terminal}

% --------------------------------------------------
\section{Création des citations}

Créez un fichier contenant des citations, blagues ou proverbes en français.

\subsection*{Travail à faire}

\begin{enumerate}
    \item Créez un fichier nommé \texttt{citations.txt}.
    \item Ajoutez au moins 5 phrases en français (une par ligne).
\end{enumerate}

\subsection*{Exemple}

\begin{terminal}
\prompt\ \shcmd{nano} citations.txt
\end{terminal}

Contenu possible :
\begin{verbatim}
Linux est puissant.
Tout est fichier.
Ça marche chez moi.
Il n'y a pas de bug, seulement des fonctionnalités.
RTFM.
\end{verbatim}

% --------------------------------------------------
\section{Afficher une citation aléatoire}

Linux fournit la commande \texttt{shuf} permettant de choisir une ligne aléatoire.

\subsection*{Travail à faire}

Affichez une citation aléatoire depuis le fichier \texttt{citations.txt}.

\begin{terminal}
\prompt\ \shcmd{shuf} \texttt{-n 1} citations.txt
\end{terminal}

% --------------------------------------------------
\section{Création d'une commande personnalisée}

Nous allons maintenant créer une commande appelée \texttt{fortune-fr}.

\subsection*{Travail à faire}

\begin{enumerate}
    \item Créez un fichier nommé \texttt{fortune-fr}.
    \item Ajoutez la ligne \texttt{\#!/bin/bash}.
    \item Écrivez un script qui affiche une citation aléatoire.
\end{enumerate}

\subsection*{Exemple de script}

\begin{terminal}
\#!/bin/bash
\shcmd{shuf} \texttt{-n 1} \$HOME/citations.txt
\end{terminal}

Rendez le script exécutable :

\begin{terminal}
\prompt\ \shcmd{chmod} \texttt{+x} fortune-fr
\end{terminal}

Testez-le :

\begin{terminal}
\prompt\ ./fortune-fr
\end{terminal}

% --------------------------------------------------
\section{Ajouter la commande au PATH}

Pour pouvoir utiliser votre commande depuis n'importe quel répertoire,
elle doit se trouver dans un répertoire du \texttt{PATH}.

\subsection*{Travail à faire}

\begin{enumerate}
    \item Créez le répertoire \texttt{\$HOME/bin} s'il n'existe pas.
    \item Déplacez votre script dans ce répertoire.
    \item Vérifiez que \texttt{\$HOME/bin} est bien dans le \texttt{PATH}.
\end{enumerate}

\begin{terminal}
\prompt\ \shcmd{mkdir} \texttt{-p} \$HOME/bin
\prompt\ \shcmd{mv} fortune-fr \$HOME/bin
\prompt\ \shcmd{echo} \$PATH
\end{terminal}

Si nécessaire, ajoutez cette ligne à votre fichier \texttt{\textasciitilde/.bashrc} :

\begin{terminal}
\shcmd{export} PATH=\$PATH:\$HOME/bin
\end{terminal}

Rechargez la configuration :

\begin{terminal}
\prompt\ \shcmd{source} \texttt{\textasciitilde/.bashrc}
\end{terminal}

% --------------------------------------------------
\section{Combiner avec cowsay}

Utilisez maintenant votre commande personnalisée avec \texttt{cowsay}.

\begin{terminal}
\prompt\ \shcmd{fortune-fr} \texttt{|} \shcmd{cowsay}
\end{terminal}

Essayez différentes exécutions.

% --------------------------------------------------
\section{Extensions (optionnelles)}

Pour aller plus loin :
\begin{itemize}
    \item ajouter une option \texttt{-h} affichant une aide ;
    \item gérer plusieurs fichiers de citations ;
    \item afficher le nombre total de citations ;
    \item utiliser une vache différente avec \texttt{cowsay -l}.
\end{itemize}

% --------------------------------------------------
\section*{Conclusion}

Vous avez créé votre propre commande Linux.
Ce principe est utilisé quotidiennement pour automatiser et personnaliser
les environnements de travail sous Unix.

\end{document}
