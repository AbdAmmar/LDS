\documentclass[11pt,a4paper]{article}

\usepackage[T1]{fontenc}
\usepackage[french]{babel}
\usepackage{geometry}

\usepackage{../tdstyle}

\geometry{margin=2.5cm}

\title{TD6 -- Langages de script}
\author{Abdallah Ammar}
\date{\today}

\begin{document}
\maketitle

% --------------------------------------------------
\section*{Objectifs du TD}

Ce TD a pour objectif de mobiliser l'ensemble des notions vues précédemment
afin de réaliser un script bash complet et utile.
À l'issue de la séance, vous devrez être capables de :
\begin{itemize}
    \item analyser une consigne complète ;
    \item écrire un script bash structuré ;
    \item automatiser des traitements sur des fichiers texte ;
    \item gérer des erreurs simples ;
    \item produire une sortie lisible.
\end{itemize}

\bigskip

\begin{warningbox}
Ce TD est un mini-projet. Prenez le temps de lire attentivement la consigne
avant d'écrire votre script.
\end{warningbox}

% --------------------------------------------------
\section{Contexte}

Vous disposez de fichiers texte contenant des informations sur des utilisateurs
et des événements (logs).  
Ces fichiers sont ceux utilisés lors du TD3.

L'objectif est d'écrire un script bash unique permettant d'analyser automatiquement
ces fichiers.

% --------------------------------------------------
\section{Consignes générales}

\begin{itemize}
    \item Le script doit s'appeler \texttt{analyze.sh}.
    \item Il doit être exécutable.
    \item Il prend un répertoire en paramètre.
    \item Il doit vérifier que ce répertoire existe.
    \item Il doit produire une sortie lisible et structurée.
    \item L'utilisation d'éditeurs graphiques est interdite.
\end{itemize}

% --------------------------------------------------
\section{Étape 1 -- Vérification des paramètres}

\subsection*{Travail à faire}

\begin{enumerate}
    \item Si aucun paramètre n'est fourni, affichez un message d'erreur.
    \item Si le paramètre fourni n'est pas un répertoire, affichez un message d'erreur.
\end{enumerate}

\subsection*{Indications}

Vous pouvez utiliser :
\begin{itemize}
    \item les paramètres du script (\texttt{\$1}) ;
    \item le test \texttt{-d}.
\end{itemize}

% --------------------------------------------------
\section{Étape 2 -- Analyse des utilisateurs}

À partir du fichier \texttt{users.txt} présent dans le répertoire fourni :

\subsection*{Travail à faire}

\begin{enumerate}
    \item Affichez le nombre total d'utilisateurs.
    \item Affichez la liste des shells utilisés.
    \item Indiquez le shell le plus utilisé.
\end{enumerate}

\subsection*{Indications}

Vous pouvez réutiliser les commandes vues en TD3 et TD5 :
\begin{itemize}
    \item \texttt{cut}
    \item \texttt{sort}
    \item \texttt{uniq}
    \item \texttt{wc}
\end{itemize}

% --------------------------------------------------
\section{Étape 3 -- Analyse des logs}

À partir du fichier \texttt{logs.txt} présent dans le répertoire fourni :

\subsection*{Travail à faire}

\begin{enumerate}
    \item Affichez le nombre total d'événements.
    \item Affichez le nombre d'erreurs.
    \item Affichez la liste des utilisateurs ayant généré une erreur.
\end{enumerate}

\subsection*{Indications}

Vous pouvez utiliser :
\begin{itemize}
    \item \texttt{grep}
    \item \texttt{cut}
    \item \texttt{sort}
    \item \texttt{uniq}
\end{itemize}

% --------------------------------------------------
\section{Étape 4 -- Affichage structuré}

Votre script doit afficher les résultats de manière claire,
avec des titres explicites pour chaque partie de l'analyse.

\subsection*{Exemple indicatif de sortie}

\begin{terminal}
=== Analyse des utilisateurs ===
Nombre total d'utilisateurs : 5
Shells utilisés :
  /bin/bash (3)
  /bin/zsh (1)
  /usr/sbin/nologin (1)

=== Analyse des logs ===
Nombre total d'événements : 20
Nombre d'erreurs : 4
Utilisateurs ayant généré une erreur :
  alice
  bob
\end{terminal}

Cet exemple est fourni à titre indicatif.
La présentation exacte peut varier.

% --------------------------------------------------
\section{Extensions (optionnelles)}

Pour les étudiants les plus rapides :

\begin{itemize}
    \item vérifier la présence des fichiers avant analyse ;
    \item sauvegarder les résultats dans un fichier ;
    \item ajouter une option \texttt{-h} affichant une aide ;
    \item analyser automatiquement tous les fichiers \texttt{.txt} du répertoire.
\end{itemize}

% --------------------------------------------------
\section*{Remarques}

\begin{itemize}
    \item Ce TD ne demande aucun concept nouveau.
    \item Réutilisez les commandes et scripts déjà écrits précédemment.
    \item Testez votre script avec différents cas.
    \item Un script clair et simple est préférable à un script complexe.
\end{itemize}

\end{document}
