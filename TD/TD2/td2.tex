\documentclass[11pt,a4paper]{article}

\usepackage[T1]{fontenc}
\usepackage[french]{babel}
\usepackage{geometry}

\usepackage{../tdstyle}

\geometry{margin=2.5cm}

\title{TD2 -- Langages de Script}
\author{Abdallah Ammar}
\date{\today}

\begin{document}
\maketitle




\begin{warningbox}
Il est fortement recommandé de taper les commandes à la main afin de se familiariser
avec le terminal.
\end{warningbox}




\bigskip

% --------------------------------------------------
\section{Commandes internes et commandes externes}

Certaines commandes sont intégrées directement dans le shell
(commandes internes ou \textit{built-in}),
tandis que d'autres sont des programmes externes.
En utilisant la commande \texttt{type} (pae exemple, \texttt{type cd}, \dots), déterminez si les commandes suivantes sont internes ou externes :
\begin{itemize}
    \item \texttt{cd}
    \item \texttt{echo}
    \item \texttt{ls}
    \item \texttt{pwd}
    \item \texttt{type}
    \item \texttt{mkdir}
    \item \texttt{useradd}
    \item \texttt{date}
    \item \texttt{time}
    \item \texttt{sleep}
    \item \texttt{touch}
    \item \texttt{cat}
    \item \texttt{grep}
\end{itemize}




% --------------------------------------------------
\section{Utiliser la documentation}

Savoir se documenter est une compétence essentielle sous Linux.

\subsection*{Travail à faire}

\begin{enumerate}
    \item Consultez la documentation de la commande \texttt{ls}.
    \item Trouvez l'option permettant d'afficher les fichiers cachés.
    \item Consultez l'aide de la commande \texttt{cd}.
    \item Expliquez pourquoi la commande \texttt{man cd} ne fonctionne pas.
    \item Consultez la documentation de \texttt{ls} avec \texttt{info}.
\end{enumerate}

\subsection*{Commandes utiles}

\begin{terminal}
\prompt\ \shcmd{man} ls

\prompt\ \shcmd{help} cd

\prompt\ \shcmd{info} ls
\end{terminal}


% --------------------------------------------------
\section{Se déplacer dans l'arborescence}

Dans cet exercice, vous allez créer une arborescence non triviale
afin de comprendre l'effet des chemins relatifs et absolus.
Après chaque changement de répertoire avec la commande \texttt{cd},
affichez systématiquement le répertoire courant à l'aide de la commande \texttt{pwd}.

\subsection*{Travail à faire}

\begin{enumerate}
    \item Allez dans votre répertoire personnel.
    \item Créez un répertoire nommé \texttt{TD2}.
    \item Entrez dans \texttt{TD2}.
    \item Créez un répertoire \texttt{TD2-bis}.
    \item Entrez dans \texttt{TD2-bis}.
    \item Créez les répertoires suivants :
    \begin{itemize}
        \item \texttt{TD2-bis-1}
        \item \texttt{TD2-bis-2}
        \item \texttt{TD2-bis-3}
    \end{itemize}
    \item Entrez dans \texttt{TD2-bis-3}.
    \item Depuis \texttt{TD2-bis-3}, créez un répertoire nommé \texttt{test}
    à l'intérieur de \texttt{TD2-bis-2}, sans vous déplacer dans ce dernier.
    \item Revenez au répertoire \texttt{TD2} avec une seule commande.
    \item Revenez ensuite dans votre répertoire personnel.
\end{enumerate}

\subsection*{Affichage de l'arborescence}

Affichez l'arborescence complète du répertoire \texttt{TD2}.

\begin{terminal}
\prompt\ \shcmd{tree} TD2
\end{terminal}

Si la commande \texttt{tree} n'est pas disponible, installez-la :

\begin{terminal}
\prompt\ \shcmd{sudo} apt update

\prompt\ \shcmd{sudo} apt install tree
\end{terminal}

% --------------------------------------------------
\section{Créer et remplir un fichier texte}

Dans cet exercice, vous allez créer un fichier texte et y écrire plusieurs lignes.

\subsection*{Travail à faire}

\begin{enumerate}
    \item Placez-vous dans le répertoire \texttt{TD2}.
    \item Créez un fichier nommé \texttt{notes.txt}.
    \item Écrivez une première ligne dans le fichier.
    \item Ajoutez deux lignes supplémentaires sans effacer les précédentes.
    \item Affichez le contenu du fichier.
\end{enumerate}

\subsection*{Commandes utiles}

\begin{terminal}
\prompt\ \shcmd{touch} notes.txt

\prompt\ \shcmd{echo} "Première ligne" \texttt{>} notes.txt

\prompt\ \shcmd{echo} "Deuxième ligne" \texttt{>>} notes.txt

\prompt\ \shcmd{echo} "Troisième ligne" \texttt{>>} notes.txt

\prompt\ \shcmd{cat} notes.txt
\end{terminal}


\section*{Pour aller plus loin -- Exercice ludique (optionnel)}

Installez et lancez le programme \texttt{sl}.
Ce programme est un clin d'œil rappelant l'importance de taper correctement
les commandes.

\begin{terminal}
\prompt\ \shcmd{sudo} apt install sl

\prompt\ \shcmd{sl}
\end{terminal}

\end{document}
