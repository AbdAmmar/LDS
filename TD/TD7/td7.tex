\documentclass[11pt,a4paper]{article}

\usepackage[T1]{fontenc}
\usepackage[french]{babel}
\usepackage{geometry}

\usepackage{../tdstyle}

\geometry{margin=2.5cm}

\title{TD7 -- Langages de script}
\author{Abdallah Ammar}
\date{\today}

\begin{document}
\maketitle

% --------------------------------------------------
\section*{Objectifs du TD}

Ce TD a pour objectif d'apprendre à analyser des fichiers de logs
similaires à ceux que l'on trouve sur un système Linux réel.
À l'issue de la séance, vous devrez être capables de :
\begin{itemize}
    \item lire et comprendre des logs système ;
    \item utiliser \texttt{grep}, \texttt{cut}, \texttt{sort}, \texttt{uniq} de manière combinée ;
    \item écrire un script bash générant un rapport lisible ;
    \item raisonner sur des données textuelles réelles.
\end{itemize}

\bigskip

\begin{warningbox}
Les fichiers utilisés dans ce TD sont fournis par l'enseignant.
N'utilisez pas directement les fichiers de \texttt{/var/log} de votre système.
\end{warningbox}

% --------------------------------------------------
\section*{Récupération des fichiers de logs}

Créez un répertoire \texttt{data} et téléchargez les fichiers de logs :

\begin{terminal}
\prompt\ \shcmd{mkdir} data
\prompt\ \shcmd{cd} data

\prompt\ \shcmd{wget} https://raw.githubusercontent.com/AbdAmmar/LDS/main/TD/TD7/data/auth.log
\prompt\ \shcmd{wget} https://raw.githubusercontent.com/AbdAmmar/LDS/main/TD/TD7/data/syslog
\prompt\ \shcmd{wget} https://raw.githubusercontent.com/AbdAmmar/LDS/main/TD/TD7/data/kernel.log
\end{terminal}

% --------------------------------------------------
\section{Lecture et observation des logs}

\subsection*{Travail à faire}

\begin{enumerate}
    \item Affichez le contenu de \texttt{auth.log}.
    \item Combien de lignes contient ce fichier ?
    \item Repérez les informations suivantes :
    \begin{itemize}
        \item date et heure ;
        \item service concerné ;
        \item message.
    \end{itemize}
\end{enumerate}

% --------------------------------------------------
\section{Recherche d'événements}

\subsection*{Travail à faire}

À partir de \texttt{auth.log} :

\begin{enumerate}
    \item Affichez les tentatives de connexion échouées.
    \item Affichez les connexions réussies.
    \item Affichez les lignes contenant \texttt{sudo}.
\end{enumerate}

\subsection*{Commandes utiles}

\begin{terminal}
\prompt\ \shcmd{grep} Failed auth.log
\prompt\ \shcmd{grep} Accepted auth.log
\prompt\ \shcmd{grep} sudo auth.log
\end{terminal}

% --------------------------------------------------
\section{Analyse par pipelines}

\subsection*{Travail à faire}

\begin{enumerate}
    \item Comptez le nombre total de tentatives de connexion échouées.
    \item Affichez la liste des utilisateurs concernés.
\end{enumerate}

\subsection*{Exemples}

\begin{terminal}
\prompt\ \shcmd{grep} Failed auth.log \texttt{|} \shcmd{wc} \texttt{-l}
\prompt\ \shcmd{grep} Failed auth.log \texttt{|} \shcmd{cut} \texttt{-d" "} \texttt{-f9} \texttt{|} \shcmd{sort} \texttt{|} \shcmd{uniq}
\end{terminal}

% --------------------------------------------------
\section{Analyse temporelle}

\subsection*{Travail à faire}

\begin{enumerate}
    \item Recherchez les événements ayant eu lieu à une heure donnée (par exemple 10h).
    \item Comptez le nombre d'événements sur cette plage horaire.
\end{enumerate}

\begin{terminal}
\prompt\ \shcmd{grep} "Jan 10 10" auth.log
\end{terminal}

% --------------------------------------------------
\section{Script : génération d'un rapport de sécurité}

\subsection*{Travail à faire}

Écrivez un script \texttt{report.sh} qui :
\begin{itemize}
    \item prend un répertoire de logs en paramètre ;
    \item analyse \texttt{auth.log} ;
    \item affiche :
    \begin{itemize}
        \item le nombre de connexions réussies ;
        \item le nombre de connexions échouées ;
        \item la liste des utilisateurs impliqués ;
    \end{itemize}
    \item produit un rapport lisible à l'écran.
\end{itemize}

\subsection*{Exemple indicatif de sortie}

\begin{terminal}
=== Rapport de sécurité ===
Connexions réussies : 3
Connexions échouées : 5

Utilisateurs concernés :
  alice
  bob
\end{terminal}

% --------------------------------------------------
\section{Extensions (optionnelles)}

\begin{itemize}
    \item analyser \texttt{syslog} ou \texttt{kernel.log} ;
    \item détecter des avertissements ;
    \item sauvegarder le rapport dans un fichier.
\end{itemize}


\section*{Pour aller plus loin -- Exercice ludique (optionnel)}

Une fois connecté à la machine distante via SSH, lancez le programme
\texttt{neofetch}.

Comparez les informations affichées avec celles de votre machine locale.

\end{document}
