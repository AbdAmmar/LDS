\documentclass[11pt,a4paper]{article}

\usepackage[T1]{fontenc}
\usepackage[french]{babel}
\usepackage{geometry}

\usepackage{../../ldsstyle}

\geometry{margin=2.5cm}

\title{TD4 -- Langages de script}
\author{Abdallah Ammar}
\date{\today}

\begin{document}
\maketitle















% --- --- --- --- --- --- --- --- --- --- --- --- --- --- --- --- --- --- --- --- --- ---


\section{Analyse de données}


\begin{enumerate}

    \item Créez le répertoire \texttt{TD4} dans votre répertoire 
          personnel (\texttt{\$HOME}), puis le sous-répertoire \texttt{Q1} 
          dans \texttt{TD4}.
          Placez-vous ensuite dans le répertoire \texttt{Q1}.

    % ---

    \item Nous allons maintenant télécharger un fichier contenant des données 
          personnelles relatives aux employés.

    \begin{terminal}
    \prompt\ \tcomment{URL du fichier à télécharger}

    \prompt\ UrlGit="https://raw.githubusercontent.com/AbdAmmar/LDS/main"

    \prompt\ Rep="TD/TD4/data/employes.csv"

    \prompt\

    \prompt\ \tcomment{Téléchargement du fichier}

    \prompt\ \shcmd{curl} -o employes.csv "\$\{UrlGit\}/\$\{Rep\}"
    
    \prompt\

    \prompt\ \tcomment{Vérifiez que le fichier a bien été téléchargé}

    \prompt\ \shcmd{ls} employes.csv
    \end{terminal}

    % ---

    \item Affichez le contenu du fichier \texttt{employes.csv} et identifiez le séparateur utilisé.

    \begin{terminal}
    \prompt\ \shcmd{cat} employes.csv
    \end{terminal}

    % ---

    \item À l'aide des commandes \texttt{cat} et \texttt{wc}, affichez le nombre de lignes
    contenues dans le fichier \texttt{employes.csv}.

    \begin{terminal}
    \prompt\ \shcmd{cat} employes.csv | \shcmd{wc} -l
    \end{terminal}

    \begin{infobox}
    Le symbole \texttt{|} (pipe) permet d'envoyer la sortie d'une commande
    directement en entrée d'une autre commande.
    Ici, la sortie de \texttt{cat} est transmise à \texttt{wc}.
    \end{infobox}

    % ---

    \item Analysez les résultats des commandes suivantes :

    \begin{terminal}
    \prompt\ \shcmd{cut} -d';' -f1 employes.csv

    \prompt\ \shcmd{cut} -d';' -f2 employes.csv

    \prompt\ \shcmd{cut} -d';' -f1,2 employes.csv

    \prompt\ \shcmd{cut} -d';' -f2,1 employes.csv

    \prompt\ \shcmd{cut} -d';' -f2,4 employes.csv

    \prompt\ \shcmd{cut} -d';' -f2,3 employes.csv

    \prompt\ \shcmd{cut} -d';' -f2,3,4 employes.csv

    \prompt\ \shcmd{cut} -d';' -f2,4,3 employes.csv

    \prompt\ \shcmd{cut} -d';' -f2-4 employes.csv

    \prompt\ \shcmd{cut} -d';' -f2- employes.csv
    \end{terminal}

    \begin{infobox}
    La commande \texttt{cut} permet d'extraire des colonnes d'un fichier texte.
    L'option \texttt{-d} indique le séparateur utilisé entre les champs,
    et l'option \texttt{-f} précise le ou les champs à afficher.
    Les champs sont numérotés à partir de \texttt{1}.
    \end{infobox}

    % ---

    \item Analysez les résultats des commandes suivantes :

    \begin{terminal}
    \prompt\ \shcmd{cut} -d';' -f1- employes.csv 

    \prompt\ \shcmd{cut} -d';' -f1- employes.csv {-}{-}output-delimiter=','

    \prompt\ \shcmd{cut} -d';' -f1- employes.csv {-}{-}output-delimiter='-'

    \prompt\ \shcmd{cut} -d';' -f1- employes.csv {-}{-}output-delimiter='*'

    \prompt\ \shcmd{cut} -d';' -f1- employes.csv {-}{-}output-delimiter=' '
    \end{terminal}

    \begin{infobox}
    L'option \texttt{--output-delimiter} permet de définir le séparateur utilisé
    pour l'affichage des champs en sortie.
    Elle ne modifie pas le fichier d'origine.
    \end{infobox}

    % ---

    \item Afin d'ignorer la ligne d'en-tête du fichier (qui commence par \texttt{id}),
    utilisez la commande \texttt{grep} pour exclure cette ligne, puis affichez
    le prénom et le nom des employés.

    \begin{terminal}
    \prompt\ \shcmd{grep} -v '\^{}id' employes.csv | \shcmd{cut} -d';' -f2,3
    \end{terminal}

    % ---

    \item Affichez uniquement le prénom et le nom des employés qui travaillent à \texttt{Paris}.

    \begin{terminal}
    \prompt\ \shcmd{grep} 'Paris' employes.csv | \shcmd{cut} -d';' -f2,3
    \end{terminal}

    % ---

    \item Analysez les résultats des commandes suivantes :

    \begin{terminal}
    \prompt\ \shcmd{grep} -v '\^{}id' employes.csv | \shcmd{cut} -d';' -f3-

    \prompt\ \shcmd{grep} -v '\^{}id' employes.csv | \shcmd{cut} -d';' -f3- | \shcmd{sort}

    \prompt\ \shcmd{grep} -v '\^{}id' employes.csv | \shcmd{cut} -d';' -f3- | \shcmd{sort} -r
    \end{terminal}

    \begin{infobox}
    Par défaut, la commande \texttt{sort} trie les lignes par ordre alphabétique.
    L'option \texttt{-r} permet d'inverser l'ordre du tri.
    \end{infobox}

    % ---

    \item Analysez les résultats des commandes suivantes :

    \begin{terminal}
    \prompt\ \shcmd{grep} -v '\^{}id' employes.csv | \shcmd{cut} -d';' -f1- | \shcmd{sort}

    \prompt\ \shcmd{grep} -v '\^{}id' employes.csv | \shcmd{cut} -d';' -f1- | \shcmd{sort} -n

    \prompt\ \shcmd{grep} -v '\^{}id' employes.csv | \shcmd{cut} -d';' -f1- | \shcmd{sort} -nr
    \end{terminal}

    \begin{infobox}
    Sans option, \texttt{sort} effectue un tri alphabétique.
    L'option \texttt{-n} permet de trier les valeurs comme des nombres.
    L'option \texttt{-r} inverse l'ordre du tri.
    \end{infobox}

    % ---

    \item Analysez les résultats des commandes suivantes :

    \begin{terminal}
    \prompt\ \shcmd{grep} -v '\^{}id' employes.csv | \shcmd{cut} -d';' -f3- | \shcmd{sort} -t';' 

    \prompt\ \shcmd{grep} -v '\^{}id' employes.csv | \shcmd{cut} -d';' -f3- | \shcmd{sort} -t';' -k1

    \prompt\ \shcmd{grep} -v '\^{}id' employes.csv | \shcmd{cut} -d';' -f3- | \shcmd{sort} -t';' -k2

    \prompt\ \shcmd{grep} -v '\^{}id' employes.csv | \shcmd{cut} -d';' -f3- | \shcmd{sort} -t';' -k3
    \end{terminal}

    \begin{infobox}
    Par défaut, \texttt{sort} trie les lignes selon le premier champ.
    L'option \texttt{-t} permet de définir le séparateur de champs,
    et l'option \texttt{-kN} permet de choisir le champ utilisé pour le tri,
    où \texttt{N} correspond au numéro du champ.
    \end{infobox}

    % ---

    \item Analysez le résultat des commandes suivantes :

    \begin{terminal}
    \prompt\ \shcmd{grep} -v '\^{}id' employes.csv | \shcmd{cut} -d';' -f4

    \prompt\ \shcmd{grep} -v '\^{}id' employes.csv | \shcmd{cut} -d';' -f4 | \shcmd{uniq}

    \prompt\ \shcmd{grep} -v '\^{}id' employes.csv | \shcmd{cut} -d';' -f4 | \shcmd{sort} | \shcmd{uniq}

    \prompt\ \shcmd{grep} -v '\^{}id' employes.csv | \shcmd{cut} -d';' -f4 | \shcmd{uniq} -c

    \prompt\ \shcmd{grep} -v '\^{}id' employes.csv | \shcmd{cut} -d';' -f4 | \shcmd{sort} | \shcmd{uniq} -c
    \end{terminal}

    \begin{infobox}
    La commande \texttt{uniq} supprime uniquement les lignes consécutives identiques.
    Si les lignes identiques ne sont pas côte à côte, elles ne seront pas supprimées.
    Pour utiliser correctement \texttt{uniq}, souvent il est nécessaire de trier les
    données au préalable à l'aide de \texttt{sort}.

    L'option \texttt{-c} de \texttt{uniq} permet d'afficher le nombre d'occurrences
    de chaque ligne.
    \end{infobox}

    % ---

    \item Combinez les commandes vues précédemment afin d'afficher les villes de l'entreprise,
    triées par nombre d'employés décroissant.

    \begin{terminal}
    \prompt\ \shcmd{grep} -v '\^{}id' employes.csv | \shcmd{cut} -d';' -f6 | \shcmd{sort} | \shcmd{uniq} -c | \shcmd{sort} -nr
    \end{terminal}

\end{enumerate}

% --- --- --- --- --- --- --- --- --- --- --- --- --- --- --- --- --- --- --- --- --- ---


% TODO
% sed pour transformer
% > pour sauvegarder
% head/tail pour exploiter



\end{document}
