\documentclass[11pt,a4paper]{article}

\usepackage[T1]{fontenc}
\usepackage[french]{babel}
\usepackage{geometry}

\usepackage{../../ldsstyle}

\geometry{margin=2.5cm}

\title{TD4 -- Langages de script}
\author{Abdallah Ammar}
\date{\today}

\begin{document}
\maketitle















% --- --- --- --- --- --- --- --- --- --- --- --- --- --- --- --- --- --- --- --- --- ---


\section{Analyse de données}


\begin{enumerate}

    \item Créez le répertoire \texttt{TD4} dans votre répertoire 
          personnel (\texttt{\$HOME}), puis le sous-répertoire \texttt{Q1} 
          dans \texttt{TD4}.
          Placez-vous ensuite dans le répertoire \texttt{Q1}.

    % ---

    \item Nous allons maintenant télécharger un fichier contenant des données 
          personnelles relatives aux employés.

    \begin{terminal}
    \prompt\ \tcomment{URL du fichier à télécharger}

    \prompt\ UrlGit="https://raw.githubusercontent.com/AbdAmmar/LDS/main"

    \prompt\ Rep="TD/TD4/data/employes.csv"

    \prompt\

    \prompt\ \tcomment{Téléchargement du fichier}

    \prompt\ \shcmd{curl} -o employes.csv "\$\{UrlGit\}/\$\{Rep\}"
    
    \prompt\

    \prompt\ \tcomment{Vérifiez que le fichier a bien été téléchargé}

    \prompt\ \shcmd{ls} employes.csv
    \end{terminal}

    % ---

    \item Affichez le contenu du fichier \texttt{employes.csv} et identifiez le séparateur utilisé.

    \begin{terminal}
    \prompt\ \shcmd{cat} employes.csv
    \end{terminal}

    % ---

    \item À l'aide des commandes \texttt{cat} et \texttt{wc}, affichez le nombre de lignes
    contenues dans le fichier \texttt{employes.csv}.

    \begin{terminal}
    \prompt\ \shcmd{cat} employes.csv | \shcmd{wc} -l
    \end{terminal}

    \begin{infobox}
    Le symbole \texttt{|} (pipe) permet d'envoyer la sortie d'une commande
    directement en entrée d'une autre commande.
    Ici, la sortie de \texttt{cat} est transmise à \texttt{wc}.
    \end{infobox}

    % ---

    \item Analysez les résultats des commandes suivantes :

    \begin{terminal}
    \prompt\ \shcmd{cut} -d';' -f1 employes.csv

    \prompt\ \shcmd{cut} -d';' -f2 employes.csv

    \prompt\ \shcmd{cut} -d';' -f1,2 employes.csv

    \prompt\ \shcmd{cut} -d';' -f2,1 employes.csv

    \prompt\ \shcmd{cut} -d';' -f2,4 employes.csv

    \prompt\ \shcmd{cut} -d';' -f2,3 employes.csv

    \prompt\ \shcmd{cut} -d';' -f2,3,4 employes.csv

    \prompt\ \shcmd{cut} -d';' -f2,4,3 employes.csv

    \prompt\ \shcmd{cut} -d';' -f2-4 employes.csv

    \prompt\ \shcmd{cut} -d';' -f2- employes.csv
    \end{terminal}

    \begin{infobox}
    La commande \texttt{cut} permet d'extraire des colonnes d'un fichier texte.
    L'option \texttt{-d} indique le séparateur utilisé entre les champs,
    et l'option \texttt{-f} précise le ou les champs à afficher.
    Les champs sont numérotés à partir de \texttt{1}.
    \end{infobox}

    % ---

    \item Analysez les résultats des commandes suivantes :

    \begin{terminal}
    \prompt\ \shcmd{cut} -d';' -f1- employes.csv 

    \prompt\ \shcmd{cut} -d';' -f1- employes.csv {-}{-}output-delimiter=','

    \prompt\ \shcmd{cut} -d';' -f1- employes.csv {-}{-}output-delimiter=' '

    \prompt\ \shcmd{cut} -d';' -f1- employes.csv {-}{-}output-delimiter='-'

    \prompt\ \shcmd{cut} -d';' -f1- employes.csv {-}{-}output-delimiter='{-}{-}'

    \prompt\ \shcmd{cut} -d';' -f1- employes.csv {-}{-}output-delimiter='{-}{+}{-}'
    \end{terminal}

    \begin{infobox}
    L'option \texttt{--output-delimiter} permet de définir le séparateur utilisé
    pour l'affichage des champs en sortie.
    Elle ne modifie pas le fichier d'origine.
    \end{infobox}

    % ---

    \item Analysez les résultats des commandes suivantes :

    \begin{terminal}
    \prompt\ \shcmd{cat} employes.csv | \shcmd{tr} ';' ' '
    
    \prompt\ \shcmd{cat} employes.csv | \shcmd{tr} ';' ','
    
    \prompt\ \shcmd{cat} employes.csv | \shcmd{tr} ';' '-'
    
    \prompt\ \shcmd{cat} employes.csv | \shcmd{tr} ';' '--'
    
    \prompt\ \shcmd{cat} employes.csv | \shcmd{tr} '[:lower:]' '[:upper:]'
    
    \prompt\ \shcmd{cat} employes.csv | \shcmd{tr} '[:upper:]' '[:lower:]'
    
    \prompt\ \shcmd{cat} employes.csv | \shcmd{tr} -d ';'
    
    \prompt\ \shcmd{cat} employes.csv | \shcmd{tr} -d '\textbackslash n'
    \end{terminal}

    \begin{infobox}
    La commande \texttt{tr} (translate) remplace ou supprime des caractères :
    \begin{itemize}
        \item \texttt{tr 'A' 'B'} : remplace chaque \texttt{A} par \texttt{B}.
        \item \texttt{tr -d 'A'} : supprime chaque \texttt{A}.
        \item \texttt{tr '[:lower:]' '[:upper:]'} : convertit les minuscules en majuscules.
    \end{itemize}
    \end{infobox}

    % ---

    \item Affichez uniquement le prénom et le nom des employés qui travaillent à \texttt{Paris}.

    \begin{terminal}
    \prompt\ \shcmd{grep} 'Paris' employes.csv | \shcmd{cut} -d';' -f2,3
    \end{terminal}

    % ---

    \item Afin d'ignorer la ligne d'en-tête du fichier (qui commence par \texttt{id}),
    utilisez la commande \texttt{grep} pour exclure cette ligne, puis affichez
    le prénom et le nom des employés.

    \begin{terminal}
    \prompt\ \shcmd{grep} -v '\^{}id' employes.csv | \shcmd{cut} -d';' -f2,3
    \end{terminal}

    % ---

    \item Générer un nouveau fichier \texttt{employes2.csv} contenant les données du 
    fichier \texttt{employes.csv}, mais sans la ligne d'en-tête de ce dernier.

    \begin{terminal}
    \prompt\ \shcmd{grep} -v '\^{}id' employes.csv > employes2.csv

    \prompt

    \prompt\ \tcomment{Vérifiez le nouveau fichier}

    \prompt\ \shcmd{cat} employes2.csv
    \end{terminal}

    \begin{infobox}
    L'opérateur \texttt{>} permet de rediriger la sortie standard d'une commande 
    vers un fichier. Si le fichier cible existe déjà, son contenu sera écrasé. 
    Pour ajouter le résultat à la fin du fichier sans l'écraser, utilisez l'opérateur \texttt{>>}.
    \end{infobox}

    % ---

    \item Analysez les résultats des commandes suivantes :

    \begin{terminal}
    \prompt\ \shcmd{cut} -d';' -f3- employes2.csv

    \prompt\ \shcmd{cut} -d';' -f3- employes2.csv | \shcmd{sort}

    \prompt\ \shcmd{cut} -d';' -f3- employes2.csv | \shcmd{sort} -r
    \end{terminal}

    \begin{infobox}
    Par défaut, la commande \texttt{sort} trie les lignes par ordre alphabétique.
    L'option \texttt{-r} permet d'inverser l'ordre du tri.
    \end{infobox}

    % ---

    \item Analysez les résultats des commandes suivantes :

    \begin{terminal}
    \prompt\ \shcmd{cut} -d';' -f1- employes2.csv | \shcmd{sort}

    \prompt\ \shcmd{cut} -d';' -f1- employes2.csv | \shcmd{sort} -n

    \prompt\ \shcmd{cut} -d';' -f1- employes2.csv | \shcmd{sort} -nr
    \end{terminal}

    \begin{infobox}
    Sans option, \texttt{sort} effectue un tri alphabétique.
    L'option \texttt{-n} permet de trier les valeurs comme des nombres.
    L'option \texttt{-r} inverse l'ordre du tri.
    \end{infobox}

    % ---

    \item Analysez les résultats des commandes suivantes :

    \begin{terminal}
    \prompt\ \shcmd{cut} -d';' -f3- employes2.csv | \shcmd{sort} -t';' 

    \prompt\ \shcmd{cut} -d';' -f3- employes2.csv | \shcmd{sort} -t';' -k1

    \prompt\ \shcmd{cut} -d';' -f3- employes2.csv | \shcmd{sort} -t';' -k2

    \prompt\ \shcmd{cut} -d';' -f3- employes2.csv | \shcmd{sort} -t';' -k3
    \end{terminal}

    \begin{infobox}
    Par défaut, \texttt{sort} trie les lignes selon le premier champ.
    L'option \texttt{-t} permet de définir le séparateur de champs,
    et l'option \texttt{-kN} permet de choisir le champ utilisé pour le tri,
    où \texttt{N} correspond au numéro du champ.
    \end{infobox}

    % ---

    \item Analysez le résultat des commandes suivantes :

    \begin{terminal}
    \prompt\ \shcmd{cut} -d';' -f4 employes2.csv

    \prompt\ \shcmd{cut} -d';' -f4 employes2.csv | \shcmd{uniq}

    \prompt\ \shcmd{cut} -d';' -f4 employes2.csv | \shcmd{sort} | \shcmd{uniq}

    \prompt\ \shcmd{cut} -d';' -f4 employes2.csv | \shcmd{uniq} -c

    \prompt\ \shcmd{cut} -d';' -f4 employes2.csv | \shcmd{sort} | \shcmd{uniq} -c
    \end{terminal}

    \begin{infobox}
    La commande \texttt{uniq} supprime uniquement les lignes consécutives identiques.
    Si les lignes identiques ne sont pas côte à côte, elles ne seront pas supprimées.
    Pour utiliser correctement \texttt{uniq}, souvent il est nécessaire de trier les
    données au préalable à l'aide de \texttt{sort}.

    L'option \texttt{-c} de \texttt{uniq} permet d'afficher le nombre d'occurrences
    de chaque ligne.
    \end{infobox}

    % ---

    \item Combinez les commandes étudiées précédemment avec \texttt{>} afin de rediriger les résultats 
    et générer les fichiers suivants :
    \begin{itemize}
        \item \texttt{employes\_id.csv} : fichier contenant les données triées par numéro d'identité ;
        \item \texttt{employes\_nom.csv} : fichier contenant les données triées par nom ;
        \item \texttt{employes\_ville.csv} : fichier contenant les données triées par nom de ville.
    \end{itemize}
    Les résultats doivent être séparés par un espace, et non par ``\texttt{;}''.

    \begin{terminal}
        \prompt\ \shcmd{cut} -d';' -f1- employes2.csv {-}{-}output-delimiter=' ' \textbackslash

        \prompt\ \hspace{0.5cm} | \shcmd{sort} -nt' ' -k1 > employes\_id.csv

        \prompt

        \prompt\ \shcmd{cut} -d';' -f1- employes2.csv {-}{-}output-delimiter=' ' \textbackslash

        \prompt\ \hspace{0.5cm} | \shcmd{sort} -t' ' -k3 > employes\_nom.csv

        \prompt

        \prompt\ \shcmd{cut} -d';' -f1- employes2.csv {-}{-}output-delimiter=' ' \textbackslash

        \prompt\ \hspace{0.5cm} | \shcmd{sort} -t' ' -k6 > employes\_ville.csv
    \end{terminal}

    \begin{infobox}
        Le caractère \texttt{\textbackslash} à la fin d'une ligne permet d'indiquer au shell que 
        la commande se poursuit sur la ligne suivante. Cela améliore la lisibilité des commandes 
        longues ou complexes en les divisant en plusieurs lignes.
    \end{infobox}

    % ---

    \item Combinez les commandes vues précédemment afin d'afficher les villes de l'entreprise,
          triées par nombre d'employés décroissant.

    \begin{terminal}
    \prompt\ \shcmd{cut} -d';' -f6 employes2.csv | \shcmd{sort} | \shcmd{uniq} -c | \shcmd{sort} -nr
    \end{terminal}

\end{enumerate}

% --- --- --- --- --- --- --- --- --- --- --- --- --- --- --- --- --- --- --- --- --- ---





\end{document}
