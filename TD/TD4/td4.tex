\documentclass[11pt,a4paper]{article}

\usepackage[T1]{fontenc}
\usepackage[french]{babel}
\usepackage{geometry}

\usepackage{../tdstyle}

\geometry{margin=2.5cm}

\title{TD4 -- Langages de script}
\author{Abdallah Ammar}
\date{\today}

\begin{document}
\maketitle

% --------------------------------------------------
\section*{Objectifs du TD}

Ce TD a pour objectif de vous initier à l'écriture de scripts bash.
À l'issue de la séance, vous devrez être capables de :
\begin{itemize}
    \item créer et exécuter un script bash ;
    \item utiliser des variables ;
    \item passer des paramètres à un script ;
    \item tester des conditions simples ;
    \item automatiser des commandes déjà connues.
\end{itemize}

\bigskip

\begin{warningbox}
Il est fortement recommandé de taper les commandes et le code à la main
afin de bien comprendre le fonctionnement des scripts.
\end{warningbox}

% --------------------------------------------------
\section{Qu'est-ce qu'un script bash ?}

Un script bash est un fichier texte contenant une suite de commandes
qui seront exécutées automatiquement par le shell.

La première ligne d'un script bash est appelée \textit{shebang} :
\texttt{\#!/bin/bash}

% --------------------------------------------------
\section{Créer et exécuter un premier script}

\subsection*{Travail à faire}

\begin{enumerate}
    \item Créez un fichier nommé \texttt{script.sh}.
    \item Ajoutez la ligne \texttt{\#!/bin/bash}.
    \item Affichez un message à l'écran.
    \item Rendez le script exécutable.
    \item Exécutez le script.
\end{enumerate}

\subsection*{Commandes utiles}

\begin{terminal}
\prompt\ \shcmd{chmod} \texttt{+x} script.sh
\prompt\ ./script.sh
\end{terminal}

% --------------------------------------------------
\section{Variables}

Les variables permettent de stocker des informations dans un script.

\subsection*{Travail à faire}

\begin{enumerate}
    \item Créez une variable contenant votre prénom.
    \item Affichez la valeur de cette variable.
    \item Modifiez la variable.
    \item Réaffichez-la.
\end{enumerate}

\subsection*{Exemple}

\begin{terminal}
prenom="Alice"
\shcmd{echo} \$prenom
\end{terminal}

% --------------------------------------------------
\section{Paramètres d'un script}

Un script peut recevoir des paramètres lors de son exécution.

\subsection*{Travail à faire}

\begin{enumerate}
    \item Modifiez votre script pour afficher le premier paramètre.
    \item Testez le script avec différents paramètres.
    \item Que contient la variable \texttt{\$0} ?
\end{enumerate}

\subsection*{Exemple}

\begin{terminal}
\shcmd{echo} \$1
\end{terminal}

% --------------------------------------------------
\section{Tests et conditions simples}

Les structures conditionnelles permettent d'exécuter du code
en fonction d'une condition.

\subsection*{Travail à faire}

\begin{enumerate}
    \item Si aucun paramètre n'est fourni, affichez un message d'erreur.
    \item Sinon, affichez le paramètre fourni.
\end{enumerate}

\subsection*{Structure générale}

\begin{terminal}
\shcmd{if} [ condition ]; \shcmd{then}
  commandes
\shcmd{fi}
\end{terminal}

% --------------------------------------------------
\section{Tester l'existence d'un fichier}

\subsection*{Travail à faire}

Écrivez un script qui :
\begin{itemize}
    \item prend un nom de fichier en paramètre ;
    \item affiche \texttt{"Fichier trouvé"} s'il existe ;
    \item affiche \texttt{"Fichier inexistant"} sinon.
\end{itemize}

\subsection*{Tests utiles}

\begin{terminal}
\texttt{-f} fichier \quad (teste si c'est un fichier)
\texttt{-d} dossier \quad (teste si c'est un répertoire)
\end{terminal}

% --------------------------------------------------
\section{Mini-script d'automatisation}

\subsection*{Travail à faire}

Écrivez un script qui :
\begin{itemize}
    \item prend un répertoire en paramètre ;
    \item vérifie qu'il existe ;
    \item affiche le nombre de fichiers qu'il contient.
\end{itemize}

\subsection*{Indications}

Vous pouvez utiliser les commandes suivantes :
\begin{itemize}
    \item \texttt{ls}
    \item \texttt{wc -l}
\end{itemize}

% --------------------------------------------------
\section*{Remarques}

\begin{itemize}
    \item Un script est simplement une suite de commandes déjà connues.
    \item Les erreurs sont normales lors de l'écriture d'un script.
    \item Lisez attentivement les messages d'erreur.
\end{itemize}

\end{document}
