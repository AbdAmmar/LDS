\documentclass[11pt,a4paper]{article}

\usepackage[T1]{fontenc}
\usepackage[french]{babel}
\usepackage{geometry}

\usepackage{../tdstyle}

\geometry{margin=2.5cm}

\title{TD5 -- Langages de script}
\author{Abdallah Ammar}
\date{\today}

\begin{document}
\maketitle

% --------------------------------------------------
\section*{Objectifs du TD}

Ce TD a pour objectif d'introduire les boucles en bash afin d'automatiser
des traitements répétitifs.
À l'issue de la séance, vous devrez être capables de :
\begin{itemize}
    \item comprendre l'intérêt des boucles ;
    \item utiliser une boucle \texttt{for} simple ;
    \item parcourir des fichiers à l'aide d'une boucle ;
    \item lire un fichier ligne par ligne ;
    \item écrire des scripts d'automatisation simples.
\end{itemize}

\bigskip

\begin{warningbox}
Il est fortement recommandé de taper le code des scripts à la main afin
de bien comprendre le fonctionnement des boucles.
\end{warningbox}

% --------------------------------------------------
\section{Pourquoi utiliser une boucle ?}

Lorsque l'on souhaite appliquer le même traitement à plusieurs éléments
(fichiers, lignes, nombres, etc.), écrire une commande par élément n'est
ni pratique ni robuste.

\subsection*{Question}

Expliquez pourquoi il n'est pas raisonnable d'écrire une commande différente
pour chaque fichier lorsque le nombre de fichiers peut varier.

% --------------------------------------------------
\section{Boucle \texttt{for} simple}

\subsection*{Travail à faire}

\begin{enumerate}
    \item Écrivez un script affichant les nombres de 1 à 5.
    \item Modifiez le script pour afficher les nombres de 1 à 10.
\end{enumerate}

\subsection*{Structure générale}

\begin{terminal}
\shcmd{for} i in 1 2 3 4 5; \shcmd{do}
  \shcmd{echo} \$i
\shcmd{done}
\end{terminal}

% --------------------------------------------------
\section{Boucle sur des fichiers}

\subsection*{Travail à faire}

Dans un répertoire contenant plusieurs fichiers texte :

\begin{enumerate}
    \item Affichez le nom de chaque fichier \texttt{.txt}.
    \item Affichez le nombre de lignes de chaque fichier.
\end{enumerate}

\subsection*{Indication}

\begin{terminal}
\shcmd{for} f in *.txt; \shcmd{do}
  \shcmd{echo} \$f
  \shcmd{wc} \texttt{-l} \$f
\shcmd{done}
\end{terminal}

% --------------------------------------------------
\section{Automatisation sur les fichiers du TD3}

Dans le TD3, vous avez manipulé plusieurs fichiers texte dans un répertoire \texttt{data}.

\subsection*{Travail à faire}

\begin{enumerate}
    \item Placez-vous dans le répertoire \texttt{data}.
    \item Parcourez tous les fichiers texte.
    \item Pour chaque fichier, affichez :
    \begin{itemize}
        \item son nom ;
        \item son nombre de lignes.
    \end{itemize}
\end{enumerate}

% --------------------------------------------------
\section{Lire un fichier ligne par ligne}

Il est possible de lire un fichier texte ligne par ligne à l'aide d'une boucle \texttt{while}.

\subsection*{Travail à faire}

\begin{enumerate}
    \item Écrivez un script qui lit le fichier \texttt{users.txt} ligne par ligne.
    \item Affichez chaque ligne précédée du texte \texttt{"Utilisateur :"}.
\end{enumerate}

\subsection*{Structure générale}

\begin{terminal}
\shcmd{while} \shcmd{read} line; \shcmd{do}
  \shcmd{echo} "Utilisateur : \$line"
\shcmd{done} \texttt{<} users.txt
\end{terminal}

% --------------------------------------------------
\section{Mini-script d'automatisation}

\subsection*{Travail à faire}

Écrivez un script qui :
\begin{itemize}
    \item prend un répertoire en paramètre ;
    \item vérifie que ce répertoire existe ;
    \item parcourt tous les fichiers \texttt{.txt} qu'il contient ;
    \item affiche pour chacun :
    \begin{itemize}
        \item le nom du fichier ;
        \item le nombre de lignes.
    \end{itemize}
\end{itemize}

\subsection*{Indications}

Vous pouvez utiliser :
\begin{itemize}
    \item les paramètres du script (\texttt{\$1}) ;
    \item le test \texttt{-d} pour vérifier l'existence d'un répertoire ;
    \item les commandes \texttt{ls} et \texttt{wc -l}.
\end{itemize}

% --------------------------------------------------
\section*{Remarques}

\begin{itemize}
    \item Une boucle permet d'éviter la répétition de code.
    \item Un script doit rester simple et lisible.
    \item En cas d'erreur, lisez attentivement les messages affichés.
\end{itemize}


\section*{Pour aller plus loin -- Exercice ludique (optionnel)}

Écrivez un script utilisant une boucle \texttt{for} pour afficher plusieurs fois
un message ou une suite de symboles à l'écran.

Par exemple :
\begin{itemize}
    \item afficher une ligne de \texttt{*}
    \item afficher les nombres de 1 à 20
\end{itemize}

\end{document}
