\documentclass[11pt,a4paper]{article}

\usepackage[T1]{fontenc}
\usepackage[french]{babel}
\usepackage{geometry}

\usepackage{../../ldsstyle}

\geometry{margin=2.5cm}

\title{TD5 -- Langages de script}
\author{Abdallah Ammar}
\date{\today}

\begin{document}
\maketitle




% https://exercism.org/tracks/bash/exercises/hamming
% https://exercism.org/tracks/bash/exercises/rna-transcription
% https://exercism.org/tracks/bash/exercises/protein-translation

% https://exercism.org/tracks/bash/exercises/crypto-square
% https://exercism.org/tracks/bash/exercises/affine-cipher
% https://exercism.org/tracks/bash/exercises/simple-cipher
% https://exercism.org/tracks/bash/exercises/atbash-cipher
% https://exercism.org/tracks/bash/exercises/rotational-cipher

% https://exercism.org/tracks/bash/exercises/roman-numerals
% https://exercism.org/tracks/bash/exercises/robot-simulator
% https://exercism.org/tracks/bash/exercises/spiral-matrix
% https://exercism.org/tracks/bash/exercises/run-length-encoding
% https://exercism.org/tracks/bash/exercises/diffie-hellman
% https://exercism.org/tracks/bash/exercises/isbn-verifier
% https://exercism.org/tracks/bash/exercises/phone-number
% https://exercism.org/tracks/bash/exercises/nucleotide-count


% https://exercism.org/tracks/bash/exercises/nth-prime






% --- --- --- --- --- --- --- --- --- --- --- --- --- --- --- --- --- --- --- --- --- ---




\section{Nombres d'Armstrong}

Un nombre d'Armstrong (ou nombre narcissique) est un nombre entier égal à la somme de 
ses propres chiffres, chacun élevé à la puissance du nombre total de ses chiffres.

Par exemple :
\begin{itemize}
    \item $9$ est un nombre d'Armstrong car $9 = 9^1 = 9$
    \item $10$ n'est pas un nombre d'Armstrong car $10 \neq 1^2 + 0^2 = 1$
    \item $153$ est un nombre d'Armstrong car $153 = 1^3 + 5^3 + 3^3 = 1 + 125 + 27 = 153$
    \item $154$ n'est pas un nombre d'Armstrong car $154 \neq 1^3 + 5^3 + 4^3 = 1 + 125 + 64 = 190$
\end{itemize}

\begin{enumerate}

    \item Créez le répertoire \texttt{TD5} dans votre répertoire 
    personnel (\texttt{\$HOME}), puis le sous-répertoire \texttt{Q1} à 
    l'intérieur de \texttt{TD5}. 
    Placez-vous ensuite dans ce répertoire \texttt{Q1}.

    % ---

    \item Créez le fichier \texttt{armstrong.sh}. Ajoutez-y 
    le \textit{Shebang} approprié et rendez le fichier exécutable.

    % ---

    \item Ajoutez une vérification au début du script pour vous assurer 
    qu'il reçoit exactement un argument. Si ce n'est pas le cas, affichez 
    un message d'erreur explicite et quittez le programme avec un code 
    de retour égal à 1.

    % ---

    \item Ajoutez un test pour vérifier que l'argument fourni est bien 
    un entier positif. Si ce n'est pas le cas, affichez un message d'erreur 
    et quittez avec un code de retour égal à 2.

    \begin{hintbox}
    Vous pouvez utiliser une expression régulière dans un test conditionnel étendu : \\
    \texttt{"\$VAR" =\textasciitilde \, \^{}[0-9]+\$}
    \end{hintbox}

    % ---

    \item Écrivez une boucle qui parcourt chaque chiffre (caractère) de 
    l'argument pour calculer la somme des chiffres élevés à la puissance 
    $n$ (où $n$ est la longueur du nombre). Affichez le résultat du calcul 
    pour vérifier.

    \begin{hintbox}
    \begin{itemize}
        \item Pour obtenir le nombre de caractères (la longueur) d'une 
        variable \texttt{VAR}, vous pouvez utiliser \texttt{\$\{\#VAR\}}.

        \item Pour extraire un caractère spécifique, vous pouvez utiliser 
        la syntaxe de sous-chaîne \texttt{\$\{VAR:start:length\}}.
    \end{itemize}
    \end{hintbox}

    % ---

    \item Complétez le script en comparant la somme calculée avec le nombre passé en argument. 
    \begin{itemize}
        \item Si les valeurs sont égales, affichez : \texttt{X est un nombre d'Armstrong}.
        \item Sinon, affichez : \texttt{X n'est pas un nombre d'Armstrong}.
    \end{itemize}

\end{enumerate}





% --- --- --- --- --- --- --- --- --- --- --- --- --- --- --- --- --- --- --- --- --- ---






\section{Le Palindrome}

Un palindrome est une chaîne de caractères (mot, phrase ou nombre) qui se lit de la même 
manière de gauche à droite et de droite à gauche. 
Dans cet exercice, nous ne tiendrons pas compte de la casse (majuscules/minuscules).

Par exemple :
\begin{itemize}
    \item \texttt{KAYAK} est un palindrome.
    \item \texttt{KAyak} sera considéré comme un palindrome (après conversion).
    \item \texttt{12321} est un palindrome.
    \item \texttt{BASH} n'est pas un palindrome (à l'envers, cela donne \texttt{HSAB}).
\end{itemize}

\begin{enumerate}

    \item Dans votre répertoire \texttt{TD5}, créez le sous-répertoire \texttt{Q2}.
    Placez-vous ensuite dans ce répertoire.

    % ---

    \item Créez le fichier \texttt{palindrome.sh}, ajoutez le \textit{Shebang} et 
    rendez le fichier exécutable.

    % ---

    \item Vérifiez que l'utilisateur a fourni exactement un argument. 
    Si ce n'est pas le cas, affichez un message d'erreur et quittez avec le code 1.

    % ---

    \item Ajoutez un test pour vérifier que l'argument est composé uniquement de 
    lettres (minuscules ou majuscules) et de chiffres. 
    Si l'argument contient d'autres caractères (ponctuation, symboles, etc.), affichez 
    une erreur et quittez avec le code 2.

    \begin{hintbox}
    Vous pouvez utiliser une expression régulière dans un test conditionnel étendu : \\
    \texttt{"\$VAR" =\textasciitilde \, \^{}[a-zA-Z0-9]+\$}
    \end{hintbox}

    % ---

    \item Avant de commencer l'inversion, convertissez l'argument en lettres minuscules 
    pour uniformiser le traitement (pour que "KAyak" devienne "kayak").

    \begin{hintbox}
    Vous pouvez utiliser la commande \texttt{tr '[:upper:]' '[:lower:]'}.
    \end{hintbox}

    % ---

    \item Créez une boucle qui parcourt les caractères de la chaîne normalisée (en minuscules) 
    pour construire une nouvelle variable contenant la chaîne inversée.

    % ---

    \item Comparez la chaîne normalisée avec la chaîne inversée.
    \begin{itemize}
        \item Si elles sont identiques, affichez : \texttt{X est un palindrome}.
        \item Sinon, affichez : \texttt{X n'est pas un palindrome}.
    \end{itemize}

\end{enumerate}







% --- --- --- --- --- --- --- --- --- --- --- --- --- --- --- --- --- --- --- --- --- ---


\end{document}
