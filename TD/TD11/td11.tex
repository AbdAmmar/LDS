\documentclass[11pt,a4paper]{article}

\usepackage[T1]{fontenc}
\usepackage[french]{babel}
\usepackage{geometry}

\usepackage{../tdstyle}

\geometry{margin=2.5cm}

\title{TD11 -- Langages de script}
\author{Abdallah Ammar}
\date{\today}

\begin{document}
\maketitle

% --------------------------------------------------
\section*{Objectifs du TD}

Ce TD a pour objectif d'apprendre à écrire des scripts bash plus lisibles,
plus structurés et plus faciles à maintenir.
À l'issue de la séance, vous devrez être capables de :
\begin{itemize}
    \item découper un script en plusieurs fonctions ;
    \item comprendre le rôle des variables dans un script ;
    \item organiser un script en parties claires ;
    \item ajouter une aide pour l'utilisateur.
\end{itemize}

\bigskip

\begin{warningbox}
Dans ce TD, l'objectif n'est pas d'écrire plus de code,
mais d'écrire du code plus clair et mieux organisé.
\end{warningbox}

% --------------------------------------------------
\section{Pourquoi structurer un script ?}

Jusqu'à présent, vous avez écrit des scripts contenant une suite de commandes.
Lorsque les scripts deviennent plus longs, il devient difficile :
\begin{itemize}
    \item de comprendre ce qu'ils font ;
    \item de les modifier ;
    \item de corriger des erreurs.
\end{itemize}

Structurer un script permet de le rendre plus lisible et plus robuste.

% --------------------------------------------------
\section{Découper un script en fonctions}

Une fonction permet de regrouper des commandes ayant un rôle précis.

\subsection*{Travail à faire}

Reprenez un script écrit lors d'un TD précédent (TD6 ou TD7 par exemple)
et identifiez les différentes parties du script.

Par exemple :
\begin{itemize}
    \item vérification des paramètres ;
    \item analyse d'un fichier ;
    \item affichage des résultats.
\end{itemize}

Créez une fonction pour chacune de ces parties.

% --------------------------------------------------
\section{Définir et utiliser des fonctions}

\subsection*{Structure générale}

\begin{terminal}
nom\_fonction() \{
  commandes
\}
\end{terminal}

Une fonction peut ensuite être appelée simplement par son nom.

\subsection*{Travail à faire}

\begin{enumerate}
    \item Créez une fonction \texttt{check\_args} qui vérifie les paramètres.
    \item Créez une fonction \texttt{analyze} qui effectue le traitement principal.
    \item Créez une fonction \texttt{print\_result} qui affiche les résultats.
    \item Appelez ces fonctions dans la partie principale du script.
\end{enumerate}

% --------------------------------------------------
\section{Variables et portée}

Les variables définies dans un script sont accessibles dans les fonctions.

\subsection*{Travail à faire}

\begin{enumerate}
    \item Définissez une variable au début du script.
    \item Modifiez cette variable dans une fonction.
    \item Affichez la variable après l'appel de la fonction.
\end{enumerate}

\subsection*{Question}

Que constatez-vous ?
Expliquez pourquoi la valeur de la variable a changé.

% --------------------------------------------------
\section{Ajouter une aide au script}

Un script bien écrit doit expliquer comment il s'utilise.

\subsection*{Travail à faire}

Ajoutez une option \texttt{-h} ou \texttt{--help} à votre script.
Lorsque cette option est fournie, le script doit afficher :
\begin{itemize}
    \item une description du script ;
    \item les paramètres attendus ;
    \item un exemple d'utilisation.
\end{itemize}

\subsection*{Indication}

Vous pouvez utiliser une structure conditionnelle du type :

\begin{terminal}
\shcmd{if} [ "\$1" = "\texttt{-h}" ]; \shcmd{then}
  \shcmd{echo} "Aide du script"
  \shcmd{exit} 0
\shcmd{fi}
\end{terminal}

% --------------------------------------------------
\section{Organisation finale du script}

Votre script final doit être organisé de la manière suivante :
\begin{itemize}
    \item commentaires en début de fichier ;
    \item définition des variables ;
    \item définition des fonctions ;
    \item partie principale (appels des fonctions).
\end{itemize}

\subsection*{Travail à faire}

Réorganisez votre script pour qu'il respecte cette structure.

% --------------------------------------------------
\section*{Remarques}

\begin{itemize}
    \item Un script lisible est plus important qu'un script court.
    \item Les fonctions permettent d'éviter les répétitions.
    \item Ce TD prépare l'écriture de scripts plus complexes.
\end{itemize}

\end{document}
