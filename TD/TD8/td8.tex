\documentclass[11pt,a4paper]{article}

\usepackage[T1]{fontenc}
\usepackage[french]{babel}
\usepackage{geometry}

\usepackage{../tdstyle}

\geometry{margin=2.5cm}

\title{TD8 -- Langages de script}
\author{Abdallah Ammar}
\date{\today}

\begin{document}
\maketitle

% --------------------------------------------------
\section*{Objectifs du TD}

Ce TD a pour objectif de découvrir l'accès distant à une machine Linux
à l'aide du protocole SSH.
À l'issue de la séance, vous devrez être capables de :
\begin{itemize}
    \item installer et démarrer un service SSH ;
    \item créer un utilisateur local ;
    \item se connecter à une machine distante ;
    \item comprendre la notion client / serveur ;
    \item travailler en binôme sur un réseau local.
\end{itemize}

\bigskip

\begin{warningbox}
Ce TD se fait en binôme.  
Suivez attentivement les consignes afin d'éviter tout problème de sécurité.
\end{warningbox}

% --------------------------------------------------
\section{Organisation du travail}

Travaillez par groupes de deux :
\begin{itemize}
    \item un étudiant joue le rôle du \textbf{serveur} ;
    \item l'autre joue le rôle du \textbf{client}.
\end{itemize}

Les rôles pourront être échangés en fin de séance.

% --------------------------------------------------
\section{Mise en place du réseau}

Pour éviter les problèmes liés au réseau de l'établissement,
utilisez le \textbf{partage de connexion d'un téléphone} afin de placer
les deux machines sur le même réseau.

Vérifiez que les deux machines sont bien connectées au même réseau.

% --------------------------------------------------
\section{Installation du service SSH}

Sur la machine \textbf{serveur} :

\begin{terminal}
\prompt\ \shcmd{sudo} apt update
\prompt\ \shcmd{sudo} apt install openssh-server
\end{terminal}

Vérifiez que le service est actif :

\begin{terminal}
\prompt\ \shcmd{systemctl} status ssh
\end{terminal}

% --------------------------------------------------
\section{Création d'un utilisateur}

Sur la machine serveur, créez un compte temporaire pour votre binôme :

\begin{terminal}
\prompt\ \shcmd{sudo} useradd \texttt{-m} invite
\prompt\ \shcmd{sudo} passwd invite
\end{terminal}

Notez le mot de passe choisi.

% --------------------------------------------------
\section{Connexion à distance}

Sur la machine \textbf{client} :

\begin{enumerate}
    \item Identifiez l'adresse IP du serveur.
    \item Connectez-vous à l'aide de SSH.
\end{enumerate}

\begin{terminal}
\prompt\ \shcmd{ssh} invite@IP\_DU\_SERVEUR
\end{terminal}

Une fois connecté :
\begin{itemize}
    \item affichez le répertoire courant ;
    \item créez un fichier ;
    \item déconnectez-vous.
\end{itemize}

\begin{terminal}
\prompt\ \shcmd{exit}
\end{terminal}

% --------------------------------------------------
\section{Nettoyage}

À la fin du TD, supprimez le compte temporaire créé :

\begin{terminal}
\prompt\ \shcmd{sudo} userdel \texttt{-r} invite
\end{terminal}

% --------------------------------------------------
\section*{Remarques}

\begin{itemize}
    \item SSH est un outil fondamental sous Linux.
    \item Ce TD ne vise pas la configuration avancée ou la sécurité fine.
    \item Les clés SSH seront vues ultérieurement ou dans un autre module.
\end{itemize}

\end{document}
