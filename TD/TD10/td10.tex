\documentclass[11pt,a4paper]{article}

\usepackage[T1]{fontenc}
\usepackage[french]{babel}
\usepackage{geometry}

\usepackage{../tdstyle}

\geometry{margin=2.5cm}

\title{TD10 -- Langages de script}
\author{Abdallah Ammar}
\date{\today}

\begin{document}
\maketitle

% --------------------------------------------------
\section*{Objectifs du TD}

Ce TD a pour objectif d'introduire la gestion des sorties et des erreurs
sous Linux.
À l'issue de la séance, vous devrez être capables de :
\begin{itemize}
    \item comprendre la différence entre sortie standard et erreur standard ;
    \item rediriger les sorties vers des fichiers ;
    \item analyser des messages d'erreur ;
    \item améliorer la robustesse de scripts bash simples.
\end{itemize}

\bigskip

\begin{warningbox}
Ce TD introduit de nouveaux concepts.
Il n'est pas attendu que tout soit parfaitement maîtrisé :
l'objectif est de comprendre les idées générales.
\end{warningbox}

% --------------------------------------------------
\section{Les flux standards}

Sous Linux, un programme communique avec l'extérieur à l'aide de flux :
\begin{itemize}
    \item l'entrée standard (stdin) ;
    \item la sortie standard (stdout) ;
    \item la sortie d'erreur (stderr).
\end{itemize}

Par défaut, la sortie standard et la sortie d'erreur sont affichées à l'écran.

% --------------------------------------------------
\section{Rediriger la sortie standard}

\subsection*{Travail à faire}

\begin{enumerate}
    \item Listez le contenu de votre répertoire personnel.
    \item Redirigez cette sortie dans un fichier \texttt{liste.txt}.
    \item Vérifiez le contenu du fichier.
\end{enumerate}

\subsection*{Commandes}

\begin{terminal}
\prompt\ \shcmd{ls}
\prompt\ \shcmd{ls} \texttt{>} liste.txt
\prompt\ \shcmd{cat} liste.txt
\end{terminal}

% --------------------------------------------------
\section{Générer et observer une erreur}

\subsection*{Travail à faire}

\begin{enumerate}
    \item Essayez d'afficher un fichier qui n'existe pas.
    \item Observez le message affiché.
\end{enumerate}

\begin{terminal}
\prompt\ \shcmd{cat} fichier\_inexistant.txt
\end{terminal}

Expliquez pourquoi le message n'est pas redirigé avec \texttt{>}.

% --------------------------------------------------
\section{Rediriger la sortie d'erreur}

La sortie d'erreur peut être redirigée séparément.

\subsection*{Travail à faire}

\begin{enumerate}
    \item Redirigez l'erreur précédente dans un fichier \texttt{erreur.txt}.
    \item Vérifiez le contenu du fichier.
\end{enumerate}

\begin{terminal}
\prompt\ \shcmd{cat} fichier\_inexistant.txt \texttt{2>} erreur.txt
\prompt\ \shcmd{cat} erreur.txt
\end{terminal}

% --------------------------------------------------
\section{Rediriger sortie et erreur}

Il est possible de rediriger à la fois la sortie standard et la sortie d'erreur.

\subsection*{Travail à faire}

\begin{enumerate}
    \item Lancez une commande produisant à la fois une sortie et une erreur.
    \item Redirigez les deux flux dans un même fichier.
\end{enumerate}

\begin{terminal}
\prompt\ \shcmd{ls} fichier\_inexistant \texttt{>} sortie.txt \texttt{2>} erreur.txt
\end{terminal}

Ou en une seule commande :

\begin{terminal}
\prompt\ \shcmd{ls} fichier\_inexistant \texttt{>} tout.txt \texttt{2>\&1}
\end{terminal}

% --------------------------------------------------
\section{Utilisation dans un script}

Les redirections sont très utiles dans les scripts pour éviter d'afficher
des messages inutiles à l'utilisateur.

\subsection*{Travail à faire}

\begin{enumerate}
    \item Reprenez un script écrit dans les TD précédents.
    \item Redirigez les messages d'erreur vers un fichier \texttt{error.log}.
    \item Vérifiez que le script continue de fonctionner.
\end{enumerate}

% --------------------------------------------------
\section{Rendre un script plus robuste}

\subsection*{Travail à faire}

Écrivez un script qui :
\begin{itemize}
    \item prend un fichier en paramètre ;
    \item vérifie qu'il existe ;
    \item affiche son contenu s'il existe ;
    \item affiche un message d'erreur clair sinon.
\end{itemize}

Comparez le comportement du script avec et sans redirections.

% --------------------------------------------------
\section*{Ouverture}

La gestion des erreurs est essentielle dans les scripts réels.
Dans des modules plus avancés, vous verrez :
\begin{itemize}
    \item les codes de retour ;
    \item la gestion des signaux ;
    \item les logs système ;
    \item les scripts robustes à grande échelle.
\end{itemize}

\end{document}
