
% ---

\begin{frame}{Le shell}
  \begin{itemize}
    \item Le shell est l'interface fondamentale du système d'exploitation
    \item Le terminal est un programme permettant d'interagir avec le shell (en ligne de commande)
    \item Les interfaces graphiques (GUI), comme \texttt{GNOME} ou \texttt{KDE}, sont optionnelles
  \end{itemize}

  \vspace{0.10 cm}
  \centering
  \begin{minipage}{\textwidth}
    \centering
    \includegraphics[width=0.5\linewidth]{images/Fig2_Barrett.png}

    %{\footnotesize\centering }
  \end{minipage}
\end{frame}

% ---

\begin{frame}{Le Shell : Interface et Langage}
  \begin{itemize}
    \item Le shell, c'est deux choses à la fois :
    \begin{itemize}
      \item \textbf{Une Interface Système :} Un programme qui lit les commandes et les transmet au noyau
      \item \textbf{Un Langage de Programmation :} Une syntaxe (variables, boucles, conditions) %permettant d'écrire des scripts pour automatiser des tâches
    \end{itemize}

    \vspace{0.3cm}

    \item \textbf{Standard et Implémentations :}
    \begin{itemize}
      \item \textbf{POSIX sh} : La norme officielle (la "grammaire") que le langage doit respecter
      \item \textbf{Les Interpréteurs (Les logiciels)} :
      \begin{itemize}
        \item \texttt{sh} : L'implémentation historique (Steve Bourne)
        \item \texttt{\textbf{bash}} (\textit{Bourne Again SHell}) : Le standard sous Linux %Il respecte la norme POSIX tout en ajoutant des outils modernes
        \item \texttt{zsh} : Populaire pour son interactivité (défaut sur macOS)
      \end{itemize}
    \end{itemize}
  \end{itemize}

  %\vspace{0.10 cm}
  %\centering
  %\begin{minipage}{\textwidth}
  %  \centering
  %  \includegraphics[width=0.49\linewidth]{images/terminal_bash.png}

  %  %{\footnotesize\centering }
  %\end{minipage}
\end{frame}

% ---

\begin{frame}{L'invite de commande (Terminal/Prompt)}
  Une fois le terminal ouvert, le shell affiche une \textbf{invite de commande} (\textit{prompt}) en attendant vos instructions

  \vspace{0.5cm}
  \begin{block}{Anatomie d'un prompt classique}
    \texttt{\textcolor{blue}{user}@\textcolor{green}{machine}:\textcolor{blue}{\textasciitilde}\$}
  \end{block}

  \begin{itemize}
    \item \texttt{\textcolor{blue}{user}} : L'utilisateur connecté
    \item \texttt{\textcolor{green}{machine}} : Le nom de l'ordinateur (hostname)
    \item \texttt{\textcolor{blue}{\textasciitilde}} : Le répertoire courant (\texttt{\textasciitilde} signifie \textit{home})
    \item \texttt{\$} : Indique un utilisateur standard (\texttt{\#} pour l'administrateur/root)
  \end{itemize}
\end{frame}

% ---

\begin{frame}{Les commandes fondamentales}
  \begin{itemize}
    \item \texttt{pwd} : affiche le chemin absolu du dossier actuel (Où suis-je ?)

    \item \texttt{ls} : affiche le contenu du dossier

    \item \texttt{cd} : permet de changer de dossier
    \begin{itemize}
      {\footnotesize
      \item \texttt{.} : dossier courant
      \item \texttt{..} : dossier parent
      \item \texttt{/} : La racine (Root)
      }
    \end{itemize}

    \item \texttt{mkdir} : crée un nouveau dossier vide

    \item \texttt{mv} : déplace un fichier OU le renomme.

    \item \texttt{cp} : copie un fichier (ou un dossier avec \texttt{-r})

    \item \texttt{rm} : Supprime un fichier \newline
          \textbf{\textcolor{red}{Attention :}} Pas de corbeille, suppression définitive !
  \end{itemize}

  %\vspace{0.2cm}
  %\centering
  %\footnotesize
  %\textit{Astuce : Si vous êtes perdu, tapez \texttt{man commande} pour lire le manuel.}
\end{frame}

% ---

\begin{frame}{Naviguer dans le système}
  \begin{columns}
    \column{0.8\textwidth}
      \begin{itemize}
        \item \texttt{\color{blue}/} : La racine (Root)
        \begin{itemize}
          \item pas de \texttt{C:} ou \texttt{D:}
          \item séparateur de dossier est \texttt{/} (et non \textbackslash)
        \end{itemize}

        \item chemin Absolu
        \begin{itemize}
            \item commence toujours par la racine \texttt{\color{blue}/}
            \item c'est l'adresse complète
        \end{itemize}

        \item Chemin Relatif
        \begin{itemize}
            \item dépend de l'endroit où l'on se trouve actuellement
            \item ne commence \textbf{pas} par \texttt{\color{blue}/}
        \end{itemize}

        \item raccourcis utiles :
        \begin{itemize}
          \item \texttt{.} : dossier courant
          \item \texttt{..} : dossier parent
        \end{itemize}
      \end{itemize}
    \column{0.2\textwidth}
    % Idéalement, insère ici une image d'arborescence (Tree)
    % \includegraphics[width=\linewidth]{images/linux_tree.png}
    \centering
    \dirtree{% Si tu utilises le package dirtree, sinon fait une liste simple
    .1 \texttt{\color{blue}/}.
    .2 \texttt{\color{blue}boot}.
    .2 \texttt{\color{blue}home}.
    .2 \texttt{\color{blue}bin}.
    .2 \texttt{\color{blue}etc}.
    .2 \texttt{\color{blue}dev}.
    .2 \texttt{\color{blue}\ldots}.
    }
  \end{columns}
\end{frame}

% ---

\begin{frame}{Vers la programmation...}
  Le Shell n'est pas seulement un lanceur de programmes, c'est aussi un langage de programmation complet (\textbf{Shell Scripting}).

  \begin{itemize}
    \item Lorsque vous tapez une commande, le Shell la lit, l'analyse et l'exécute immédiatement.
    \item C'est un fonctionnement \textbf{Interprété}.
  \end{itemize}

  \vspace{0.5cm}
  \begin{alertblock}{Question}
    Quelle est la différence avec des langages comme C ou C++ que nous verrons plus tard (ou Java) ?
    \rightarrow \text{Introduction à la compilation vs interprétation.}
  \end{alertblock}
\end{frame}

%\begin{frame}{TODO}
%  \begin{itemize}
%    \item Commande \textit{built-in} (ou \textit{keyword}) vs commande externe
%    \item An executable program (in /usr/bin): binaires ou scripts (ls, cp, which)
%    \item A command built into the shell itself (type, cd, help)
%    \item A shell function
%    \item An alias
%  \end{itemize}
%\end{frame}

% ---

%\begin{frame}{Premières commandes du shell}
%  \begin{itemize}
%    \item Commande \textit{built-in} (ou \textit{keyword}) vs commande externe
%    \item \texttt{pwd} : affiche le répertoire courant
%    \item \texttt{ls} : liste le contenu d'un répertoire
%    \item \texttt{cd} : change de répertoire
%    \item \texttt{mkdir} : crée un répertoire
%    \item \texttt{touch} : crée un fichier vide
%  \end{itemize}
%\end{frame}

% ---

%\begin{frame}{Arborescence et chemins}
%  \begin{itemize}
%    \item Le système de fichiers est organisé en arborescence
%    \item La racine est notée \texttt{/}
%    \item Deux types de chemins :
%    \begin{itemize}
%      \footnotesize
%      \item Chemin absolu : commence par \texttt{/}
%      \item Chemin relatif : dépend du répertoire courant
%    \end{itemize}
%    \item Raccourcis :
%    \begin{itemize}
%      \footnotesize
%      \item \texttt{.} : répertoire courant
%      \item \texttt{..} : répertoire parent
%    \end{itemize}
%  \end{itemize}
%\end{frame}
%
%% ---
%
%\begin{frame}{Fichiers et permissions}
%  \begin{itemize}
%    \item Chaque fichier possède des permissions :
%    \begin{itemize}
%      \footnotesize
%      \item \texttt{r} : lecture
%      \item \texttt{w} : écriture
%      \item \texttt{x} : exécution
%    \end{itemize}
%    \item Trois niveaux :
%    \begin{itemize}
%      \footnotesize
%      \item utilisateur (u), groupe (g), autres (o)
%    \end{itemize}
%    \item Commandes utiles :
%    \begin{itemize}
%      \footnotesize
%      \item \texttt{ls -l}
%      \item \texttt{chmod}
%    \end{itemize}
%  \end{itemize}
%\end{frame}
%
%% ---
%
%\begin{frame}{Redirections et pipes}
%
%\begin{itemize}
%  \item Entrée / sortie standard :
%  \begin{itemize}
%    \footnotesize
%    \item \texttt{stdin}, \texttt{stdout}, \texttt{stderr}
%  \end{itemize}
%  \item Redirections :
%  \begin{itemize}
%    \footnotesize
%    \item \texttt{>} : redirige la sortie
%    \item \texttt{<} : redirige l'entrée
%  \end{itemize}
%  \item Pipe :
%  \begin{itemize}
%    \footnotesize
%    \item \texttt{|} : relie la sortie d'un programme à l'entrée d'un autre
%  \end{itemize}
%\end{itemize}
%
%\end{frame}
%
%% ---
%
%\begin{frame}{Processus et contrôle}
%  \begin{itemize}
%    \item Une commande lancée = un processus
%    \item Commandes utiles :
%    \begin{itemize}
%      \footnotesize
%      \item \texttt{ps} : liste des processus
%      \item \texttt{top} : processus en temps réel
%      \item \texttt{kill} : termine un processus
%    \end{itemize}
%    \item Exécution en arrière-plan :
%    \begin{itemize}
%      \footnotesize
%      \item \texttt{\&}
%    \end{itemize}
%  \end{itemize}
%\end{frame}
%
%% ---
%
