
% ---

\begin{frame}{Le shell}
  \begin{itemize}
    \item Le shell est l'interface fondamentale du système d'exploitation
    \item Le terminal est un programme permettant d'interagir avec le shell (en ligne de commande)
    \item Les interfaces graphiques (GUI), comme \texttt{GNOME} ou \texttt{KDE}, sont optionnelles
  \end{itemize}

  \vspace{0.10 cm}
  \centering
  \begin{minipage}{\textwidth}
    \centering
    \includegraphics[width=0.5\linewidth]{images/Fig2_Barrett.png}

    %{\footnotesize\centering }
  \end{minipage}
\end{frame}

% ---

\begin{frame}{Le Shell}
  \begin{itemize}
    \setlength\itemsep{1em}

    \item Le shell
    \begin{itemize}
      \item interface système : un programme qui lit les commandes et les transmet au noyau
      \item langage de programmation : une syntaxe (variables, boucles, conditions)
    \end{itemize}

    \pause

    \item Interpréteurs (logiciels)
    \begin{itemize}
      \setlength\itemsep{0.5em}
      \footnotesize
      \item \texttt{sh} : l'implémentation historique (Steve Bourne)
      \item \texttt{bash} : \textit{Bourne Again SHell} (défaut sous Linux)
      \item \texttt{zsh} : Populaire pour son interactivité (défaut sur macOS)
    \end{itemize}
  \end{itemize}

\end{frame}

% ---

\begin{frame}{L'invite de commande (Terminal/Prompt)}

  \begin{block}{Anatomie d'un prompt classique}
    \texttt{\textcolor{blue}{user}@\textcolor{green}{machine}:\textcolor{blue}{\textasciitilde}\$}
  \end{block}

  \begin{itemize}
    \setlength\itemsep{0.5em}

    \item \texttt{\textcolor{blue}{user}} : l'utilisateur connecté
    \item \texttt{\textcolor{green}{machine}} : le nom de l'ordinateur (hostname)
    \item \texttt{\textcolor{blue}{\textasciitilde}} : le répertoire courant (\texttt{\textasciitilde} signifie \textit{home})
    \item \texttt{\$} : indique un utilisateur standard (\texttt{\#} pour l'administrateur/root)
  \end{itemize}

  \pause

  \vspace{0.3 cm}
  \centering
  \begin{minipage}{\textwidth}
    \centering
    \includegraphics[width=0.6\linewidth]{images/terminal_env.png}

    %{\footnotesize\centering }
  \end{minipage}

\end{frame}

% ---

\begin{frame}{Naviguer dans le système}

  \begin{columns}

    \column{0.8\textwidth}
      \begin{itemize}
        \setlength\itemsep{1em}

        \onslide<2->{
        \item la racine (Root) : \texttt{\color{blue}/}
        \begin{itemize}
          \footnotesize
          \setlength\itemsep{0.5em}
          \item pas de \texttt{C:} ou \texttt{D:}
          \item séparateur de dossier est \texttt{/} (et non \textbackslash)
        \end{itemize}
        }

        \onslide<3->{
        \item chemin absolu
        \begin{itemize}
          \footnotesize
          \setlength\itemsep{0.5em}
          \item commence toujours par la racine \texttt{\color{blue}/}
          \item c'est l'adresse complète
        \end{itemize}
        }

        \onslide<4->{
        \item chemin relatif
        \begin{itemize}
          \footnotesize
          \setlength\itemsep{0.5em}
          \item dépend de l'endroit où l'on se trouve actuellement
          \item ne commence pas par \texttt{\color{blue}/}
        \end{itemize}
        }

        \onslide<5->{
        \item raccourcis utiles :
        \begin{itemize}
          \footnotesize
          \setlength\itemsep{0.5em}
          \item dossier courant : \texttt{\color{blue}.}
          \item dossier parent : \texttt{\color{blue}..}
        \end{itemize}
        }
      \end{itemize}

    \column{0.2\textwidth}

      \onslide<1->{
      \centering
      \dirtree{%
      .1 \texttt{\color{blue}/}.
      .2 \texttt{\color{blue}boot}.
      .2 \texttt{\color{blue}home}.
      .2 \texttt{\color{blue}bin}.
      .2 \texttt{\color{blue}etc}.
      .2 \texttt{\color{blue}dev}.
      .2 \texttt{\color{blue}\ldots}.
      }
      }
  \end{columns}
\end{frame}

% ---

\begin{frame}{Les commandes fondamentales}

  \begin{itemize}
    \setlength\itemsep{1em}

    \item \texttt{pwd} : affiche le chemin absolu du dossier actuel (où suis-je ?)

    \pause

    \item \texttt{ls} : affiche le contenu du dossier

    \pause

    \item \texttt{cd} : change de dossier

    \pause

    \item \texttt{mkdir} : crée un nouveau dossier vide

    \pause

    \item \texttt{mv} : déplace un fichier ou le renomme

    \pause

    \item \texttt{cp} : copie un fichier (ou un dossier avec \texttt{-r})

    \pause

    \item \texttt{rm} : supprime un fichier (ou un dossier avec \texttt{-r}) \newline
          \emoji{warning} pas de corbeille, suppression définitive !
  \end{itemize}

\end{frame}

% ---

%\begin{frame}{Fichiers et permissions}
%  \begin{itemize}
%    \item Chaque fichier possède des permissions :
%    \begin{itemize}
%      \footnotesize
%      \item \texttt{r} : lecture
%      \item \texttt{w} : écriture
%      \item \texttt{x} : exécution
%    \end{itemize}
%    \item Trois niveaux :
%    \begin{itemize}
%      \footnotesize
%      \item utilisateur (u), groupe (g), autres (o)
%    \end{itemize}
%    \item Commandes utiles :
%    \begin{itemize}
%      \footnotesize
%      \item \texttt{ls -l}
%      \item \texttt{chmod}
%    \end{itemize}
%  \end{itemize}
%\end{frame}
%
%% ---
%
%\begin{frame}{Processus et contrôle}
%  \begin{itemize}
%    \item Une commande lancée = un processus
%    \item Commandes utiles :
%    \begin{itemize}
%      \footnotesize
%      \item \texttt{ps} : liste des processus
%      \item \texttt{top} : processus en temps réel
%      \item \texttt{kill} : termine un processus
%    \end{itemize}
%    \item Exécution en arrière-plan :
%    \begin{itemize}
%      \footnotesize
%      \item \texttt{\&}
%    \end{itemize}
%  \end{itemize}
%\end{frame}
%
%% ---
%
