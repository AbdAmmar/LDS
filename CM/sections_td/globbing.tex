
% ---

% --- Diapositive 1 : Introduction ---
\begin{frame}{Globbing}
  \begin{itemize}
    \item Le \textbf{globbing} (ou "expansion des noms de fichiers") est un mécanisme du shell qui permet de faire 
          correspondre des noms de fichiers en utilisant des caractères spéciaux appelés \textbf{jokers} (ou \textit{wildcards} en anglais).
  
    \item L'expansion est faite par le \textbf{shell} \textit{avant} que la commande ne soit exécutée.
  
    \item Par exemple,
    \begin{itemize}
      \item Quand vous tapez \texttt{ls *.txt}, le shell cherche tous les fichiers qui correspondent à \texttt{*.txt}.
      \item Il remplace ensuite \texttt{*.txt} par la liste des fichiers trouvés.
      \item La commande \texttt{ls} ne voit que la liste finale des fichiers.
    \end{itemize}
  \end{itemize}
\end{frame}

% ---

% --- Diapositive 2 : Le joker * ---
\begin{frame}{Le Joker le plus courant : \texttt{*} (l'astérisque)}
  
  \begin{block}{Règle}
    L'astérisque \texttt{*} correspond à \textbf{n'importe quelle chaîne de caractères}, y compris une chaîne vide (zéro ou plusieurs caractères).
  \end{block}
  
  \begin{exampleblock}{Exemples}
    Imaginons les fichiers suivants dans un dossier : \\
    \texttt{rapport.txt rapport\_final.pdf image.jpg doc.txt}
    
    \begin{itemize}
      \item \texttt{ls *.txt} \\
      \textit{Résultat :} \texttt{rapport.txt doc.txt}
      \item \texttt{ls rap*} \\
      \textit{Résultat :} \texttt{rapport.txt rapport\_final.pdf}
      \item \texttt{ls *} \\
      \textit{Résultat :} \texttt{rapport.txt rapport\_final.pdf image.jpg doc.txt}
    \end{itemize}
  \end{exampleblock}
\end{frame}

% ---

% --- Diapositive 3 : Le joker ? ---
\begin{frame}
  \frametitle{Le Joker précis : \texttt{?} (le point d'interrogation)}
  
  \begin{block}{Règle}
    Le point d'interrogation \texttt{?} correspond à \textbf{exactement un} caractère, quel qu'il soit.
  \end{block}
  
  \begin{exampleblock}{Exemples}
    Imaginons les fichiers : \\
    \texttt{photo1.jpg photo2.jpg photoA.jpg photo10.jpg}
    
    \begin{itemize}
      \item \texttt{ls photo?.jpg} \\
      \textit{Résultat :} \texttt{photo1.jpg photo2.jpg photoA.jpg}
      \item \textbf{Pourquoi pas \texttt{photo10.jpg} ?} \\
      Car \texttt{?} ne remplace qu'un seul caractère, alors que "10" en contient deux.
      \item \texttt{ls photo??.jpg} \\
      \textit{Résultat :} \texttt{photo10.jpg}
    \end{itemize}
  \end{exampleblock}
\end{frame}

% --- Diapositive 4 : Les crochets [] ---
\begin{frame}{Les ensembles de caractères : \texttt{[...]} (les crochets)}

  \begin{block}{Règle}
    Les crochets \texttt{[...]} correspondent à \textbf{un seul caractère} parmi ceux spécifiés dans l'ensemble. On peut y mettre une liste ou un intervalle.
  \end{block}
  
  \begin{exampleblock}{Exemples}
    Imaginons les fichiers : \\
    \texttt{rapport1.pdf rapport2.pdf rapport3.pdf rapportA.pdf}
    
    \begin{itemize}
      \item \textbf{Avec une liste :} \texttt{ls rapport[13A].pdf} \\
      \textit{Résultat :} \texttt{rapport1.pdf rapport3.pdf rapportA.pdf}
      \item \textbf{Avec un intervalle :} \texttt{ls rapport[1-3].pdf} \\
      \textit{Résultat :} \texttt{rapport1.pdf rapport2.pdf rapport3.pdf}
      \item \textbf{Intervalle alphabétique :} \texttt{ls fichier[a-c].log} \\
      \textit{Correspondra à :} \texttt{fichiera.log}, \texttt{fichierb.log}, \texttt{fichierc.log}
    \end{itemize}
  \end{exampleblock}
\end{frame}

% --- Diapositive 5 : Combinaisons ---
\begin{frame}{La puissance des combinaisons}
  \begin{exampleblock}{Exemples concrets}
    \begin{itemize}
      \item Lister toutes les images (\texttt{.jpg}, \texttt{.png}, \texttt{.gif}) qui commencent par "vacances" et ont un numéro à deux chiffres :
      \begin{terminal}
        \prompt\ \shcmd{ls} vacances[0-9][0-9].*
      \end{terminal}
      \textit{Correspondra à \texttt{vacances01.jpg}, \texttt{vacances25.png}, mais pas \texttt{vacances1.jpg}}.
      
      \item Supprimer tous les fichiers de log (\texttt{.log}) sauf celui du jour 5 :
      \begin{terminal}
        \prompt\ \shcmd{rm} 2023-10-[0-4]*.log 2023-10-[6-9]*.log
      \end{terminal}
      
      \item Lister les chapitres 1 à 9 de n'importe quel type de fichier :
      \begin{terminal}
        \prompt\ \shcmd{ls} chap?.*
      \end{terminal}
      \textit{Ou plus précisément :}
      \begin{terminal}
        \prompt\ \shcmd{ls} chap[1-9].*
      \end{terminal}
    \end{itemize}
  \end{exampleblock}
\end{frame}

% --- Diapositive 6 : Négation et classes ---
%\begin{frame}
%  \frametitle{Aller plus loin : Négation et Classes de Caractères}
%  
%  \begin{block}{La Négation : \texttt{[!...]} ou \texttt{[^...]}}
%    Le point d'exclamation (ou l'accent circonflexe) en premier dans les crochets signifie "tout sauf".
%    
%    \begin{itemize}
%      \item \texttt{ls photo[!0-9].jpg} \\
%      \textit{Liste les fichiers \texttt{photoX.jpg} où X n'est pas un chiffre.}
%    \end{itemize}
%  \end{block}
%  
%  \pause
%  
%  \begin{block}{Les Classes de Caractères POSIX}
%    Pour plus de portabilité, on peut utiliser des classes prédéfinies.
%    \begin{itemize}
%        \item \texttt{[[:alpha:]]} : Toutes les lettres (a-z, A-Z)
%        \item \texttt{[[:digit:]]} : Tous les chiffres (0-9)
%        \item \texttt{[[:alnum:]]} : Tous les alphanumériques
%        \item \texttt{[[:lower:]]} : Toutes les lettres minuscules
%    \end{itemize}
%    \textbf{Exemple :} \texttt{ls rapport[[:digit:]].pdf} est équivalent à \texttt{ls rapport[0-9].pdf}.
%  \end{block}
%\end{frame}
%
%
%% --- Diapositive 7 : Globbing vs Regex ---
%\begin{frame}
%  \frametitle{Globbing vs. Expressions Régulières (Regex)}
%  
%  C'est une confusion très fréquente !
%  
%  \begin{columns}[T]
%    \begin{column}{.5\textwidth}
%      \begin{block}{Globbing}
%        \begin{itemize}
%          \item \textbf{But :} Expansion des noms de fichiers.
%          \item \textbf{Où :} Dans le shell (\texttt{ls}, \texttt{cp}, \texttt{mv}...).
%          \item \textbf{Syntaxe :} Simple (\texttt{*}, \texttt{?}, \texttt{[]}).
%          \item \texttt{*} = Zéro ou plusieurs caractères.
%        \end{itemize}
%      \end{block}
%    \end{column}
%    \begin{column}{.5\textwidth}
%      \begin{block}{Expressions Régulières (Regex)}
%        \begin{itemize}
%          \item \textbf{But :} Recherche de motifs dans du texte.
%          \item \textbf{Où :} Dans des commandes comme \texttt{grep}, \texttt{sed}, \texttt{awk}, et en programmation.
%          \item \textbf{Syntaxe :} Plus riche et complexe (\texttt{.}, \texttt{+}, \texttt{*}, \texttt{^}, \texttt{\$}, \texttt{()}).
%          \item \texttt{*} = Zéro ou plusieurs occurrences du caractère \textit{précédent}.
%        \end{itemize}
%      \end{block}
%    \end{column}
%  \end{columns}
%\end{frame}
%
%% ---
%
%% --- Diapositive 8 : Résumé ---
%\begin{frame}
%  \frametitle{Résumé}
%  
%  \begin{itemize}
%    \item Le \textbf{globbing} est une fonctionnalité du shell pour manipuler des groupes de fichiers.
%    \item Les jokers principaux sont :
%    \begin{itemize}
%      \item \texttt{*} : Zéro ou plusieurs caractères quelconques.
%      \item \texttt{?} : Exactement un caractère quelconque.
%      \item \texttt{[...]} : Exactement un caractère parmi un ensemble ou un intervalle.
%      \item \texttt{[!...]} : Exactement un caractère qui n'est pas dans l'ensemble.
%    \end{itemize}
%    \item La combinaison de ces jokers vous permet de cibler des fichiers de manière très précise et efficace.
%    \item C'est un outil fondamental pour travailler en ligne de commande.
%  \end{itemize}
%\end{frame}

% ---

