
% ---

\begin{frame}{Globbing (expansion des noms de fichiers)}
  \begin{itemize}
    \item Le \texttt{globbing} est un mécanisme du shell qui permet de faire 
          correspondre des \texttt{motifs} (\emph{patterns}) en utilisant des 
          caractères spéciaux appelés \texttt{jokers} (\emph{wildcards})
  
    \item L'expansion est faite par le shell \textbf{avant} que la commande ne soit exécutée
    \begin{terminal}
      \prompt\ \shcmd{ls} *.txt
    \end{terminal}
    \begin{itemize}
      \pause
      \item Le shell cherche tous les fichiers qui correspondent à \texttt{*.txt}
      \pause
      \item Il remplace ensuite \texttt{*.txt} par la liste des fichiers trouvés
      \pause
      \item La commande \texttt{ls} ne voit que la liste finale des fichiers
    \end{itemize}

    \pause

    \item Les jokers principaux sont :
    \begin{itemize}
      \item \texttt{*} : Zéro ou plusieurs caractères quelconques
      \pause
      \item \texttt{?} : Exactement un caractère quelconque
      \pause
      \item \texttt{[...]} : Exactement un caractère parmi un ensemble ou un intervalle
      \pause
      \item \texttt{[!...]} : Exactement un caractère qui n'est pas dans l'ensemble
    \end{itemize}
  \end{itemize}
\end{frame}

% ---

\begin{frame}{L'astérisque : \texttt{*}}
  \begin{itemize}
    \item L'astérisque \texttt{*} correspond à \textbf{n'importe quelle chaîne de caractères}, 
          y compris une chaîne vide (zéro ou plusieurs caractères)

    \pause

    \item Imaginons les fichiers suivants dans un dossier :
    \begin{terminal}
    \only<1,2,5,8>{
    report.txt report\_final.pdf image.jpg doc.txt
    }

    \only<3,4>{
    report{\color{red} .txt} report\_final.pdf image.jpg doc{\color{red} .txt}
    }

    \only<6,7>{
    {\color{red} rap}port.txt {\color{red} rap}port\_final.pdf image.jpg doc.txt
    }

    \only<9,10>{
    {\color{red}report.txt report\_final.pdf image.jpg doc.txt}
    }
    \end{terminal}

    \begin{terminal}
    \onslide<2->{
    \prompt\ \shcmd{ls} *.txt
    }

    \onslide<4->{
    report.txt doc.txt
    }
    \end{terminal}

    \begin{terminal}
    \onslide<5->{
    \prompt\ \shcmd{ls} rap*
    }

    \onslide<7->{
    report.txt report\_final.pdf
    }
    \end{terminal}

    \begin{terminal}
    \onslide<8->{
    \prompt\ \shcmd{ls} *
    }

    \onslide<10->{
    report.txt report\_final.pdf image.jpg doc.txt
    }
    \end{terminal}
  \end{itemize}
\end{frame}

% ---

\begin{frame}{Le point d'interrogation: \texttt{?}}
  \begin{itemize}
    \item Le point d'interrogation \texttt{?} correspond à \textbf{exactement un caractère}

    \pause

    \item Imaginons les fichiers :
    \begin{terminal}
    \only<1,2,5>{
    photo1.jpg photo2.jpg photoA.jpg photo10.jpg
    }

    \only<3,4>{
    {\color{red} photo}1{\color{red} .jpg} 
    {\color{red} photo}2{\color{red} .jpg}
    {\color{red} photo}A{\color{red} .jpg} 
    photo10.jpg
    }

    \only<6,7>{
    photo1.jpg photo2.jpg photoA.jpg {\color{red} photo}10{\color{red}.jpg}
    }
    \end{terminal}

    \begin{terminal}
    \onslide<2->{
    \prompt\ \shcmd{ls} photo?.jpg
    }

    \onslide<4->{
    photo1.jpg photo2.jpg photoA.jpg
    }
    \end{terminal}

    \begin{terminal}
    \onslide<5->{
    \prompt\ \shcmd{ls} photo??.jpg
    }

    \onslide<7->{
    photo10.jpg
    }
    \end{terminal}
  \end{itemize}
\end{frame}

% ---

\begin{frame}{Les crochets : \texttt{[...]}}
  \begin{itemize}
    \item Les crochets \texttt{[...]} correspondent à \textbf{un seul caractère} parmi 
          ceux spécifiés dans l'ensemble

    \pause
  
    \item Imaginons les fichiers :
    \begin{terminal}
    \only<1,2,5,8>{
    rep1.pdf rep2.pdf rep3.pdf repA.pdf repB.pdf
    }

    \only<3,4>{
    {\color{red} rep}1{\color{red} .pdf}
    {\color{red} rep}2{\color{red} .pdf}
    {\color{red} rep}3{\color{red} .pdf}
    {\color{red} rep}A{\color{red} .pdf}
    {\color{red} rep}B{\color{red} .pdf}
    }

    \only<6,7>{
    {\color{red} rep}1{\color{red} .pdf}
    {\color{red} rep}2{\color{red} .pdf}
    {\color{red} rep}3{\color{red} .pdf}
    {\color{red} rep}A{\color{red} .pdf}
    {\color{red} rep}B{\color{red} .pdf}
    }

    \only<9,10>{
    {\color{red} rep}1{\color{red} .pdf}
    {\color{red} rep}2{\color{red} .pdf}
    {\color{red} rep}3{\color{red} .pdf}
    {\color{red} rep}A{\color{red} .pdf}
    {\color{red} rep}B{\color{red} .pdf}
    }
    \end{terminal}

    \begin{terminal}
    \onslide<2->{
    \prompt\ \shcmd{ls} rep[13A].pdf
    }

    \onslide<4->{
    rep1.pdf rep3.pdf repA.pdf
    }
    \end{terminal}

    \begin{terminal}
    \onslide<5->{
    \prompt\ \shcmd{ls} rep[0-9].pdf
    }

    \onslide<7->{
    rep1.pdf rep2.pdf rep3.pdf
    }
    \end{terminal}

    \begin{terminal}
    \onslide<8->{
    \prompt\ \shcmd{ls} rep[A-Z].pdf
    }

    \onslide<10->{
    repA.pdf repB.pdf
    }
    \end{terminal}
  \end{itemize}
\end{frame}

% ---

\begin{frame}{La Négation \texttt{[!...]}}
  \begin{itemize}
    \item Le point d'exclamation ! (ou \^{}) placé en \textbf{première position} dans 
          les crochets correspond à \textbf{un seul caractère} qui n'est \textit{pas} 
          dans la liste ou l'intervalle spécifié
  
    \pause

    \item Imaginons les fichiers suivants :
    \begin{terminal}
    \only<1,2,5,8>{
    rapport1.log rapportA.log rapportB.log rapport\_final.log
    }

    \only<3,4>{
    {\color{red}rapport}1.log {\color{red}rapport}A.log {\color{red}rapport}B.log {\color{red}rapport}\_final.log
    }

    \only<6,7>{
    {\color{red}r}apport1.log {\color{red}r}apportA.log {\color{red}r}apportB.log {\color{red}r}apport\_final.log
    }

    \only<9,10>{
    {\color{red}rapport}1.log {\color{red}rapport}A.log {\color{red}rapport}B.log {\color{red}rapport}\_final.log
    }
    \end{terminal}

    \begin{terminal}
    \onslide<2->{
    \prompt\ \shcmd{ls} rapport[!0-9].log
    }

    \onslide<4->{
    rapportA.log rapportB.log rapport\_final.log
    }
    \end{terminal}

    \begin{terminal}
    \onslide<5->{
    \prompt\ \shcmd{ls} [!abc]*
    }

    \onslide<7->{
    rapport1.log rapportA.log rapportB.log rapport\_final.log
    }
    \end{terminal}

    \begin{terminal}
    \onslide<8->{
    \prompt\ \shcmd{ls} rapport[!A-Z].log
    }

    \onslide<10->{
    rapport1.log rapport\_final.log
    }
    \end{terminal}
  \end{itemize}
\end{frame}

% ---

% --- Diapositive 7 : Globbing vs Regex ---
\begin{frame}{Globbing vs. Expressions Régulières (Regex)}
  
  C'est une confusion très fréquente !
  
  \begin{columns}[T]
    \begin{column}{.5\textwidth}
      \begin{block}{Globbing}
        \begin{itemize}
          \item \textbf{But :} Expansion des noms de fichiers.
          \item \textbf{Où :} Dans le shell (\texttt{ls}, \texttt{cp}, \texttt{mv}...).
          \item \textbf{Syntaxe :} Simple (\texttt{*}, \texttt{?}, \texttt{[]}).
          \item \texttt{*} = Zéro ou plusieurs caractères.
        \end{itemize}
     \end{block}
    \end{column}
    \begin{column}{.5\textwidth}
      \begin{block}{Expressions Régulières (Regex)}
        \begin{itemize}
          \item \textbf{But :} Recherche de motifs dans du texte.
          \item \textbf{Où :} Dans des commandes comme \texttt{grep}, \texttt{sed}, \texttt{awk}, et en programmation.
          \item \textbf{Syntaxe :} Plus riche et complexe (\texttt{.}, \texttt{+}, \texttt{*}, \texttt{\^{}}, \texttt{\$}, \texttt{()}).
          \item \texttt{*} = Zéro ou plusieurs occurrences du caractère \textit{précédent}.
        \end{itemize}
      \end{block}
    \end{column}
  \end{columns}
\end{frame}

% ---

% ---

