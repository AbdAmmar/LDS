
% ---

\begin{frame}{Globbing (expansion des noms de fichiers)}

  \begin{itemize}
    \setlength\itemsep{1em}

    \item le \texttt{globbing} est un mécanisme du shell qui permet de faire 
          correspondre des \texttt{motifs} (\emph{patterns}) en utilisant des 
          caractères spéciaux appelés \texttt{jokers} (\emph{wildcards})

    \pause
  
    \item l'expansion est faite par le shell \textbf{avant} que la commande ne soit exécutée
    \begin{terminal}
      \prompt\ \shcmd{ls} *.txt
    \end{terminal}
    \begin{itemize}
      \item le shell cherche tous les fichiers qui correspondent à \texttt{*.txt}
      \item il remplace ensuite \texttt{*.txt} par la liste des fichiers trouvés
      \item la commande \texttt{ls} ne voit que la liste finale des fichiers
    \end{itemize}

%    \pause
%
%    \item Les jokers principaux sont :
%    \begin{itemize}
%      \item \texttt{*} : Zéro ou plusieurs caractères quelconques
%      \pause
%      \item \texttt{?} : Exactement un caractère quelconque
%      \pause
%      \item \texttt{[...]} : Exactement un caractère parmi un ensemble ou un intervalle
%      \pause
%      \item \texttt{[!...]} : Exactement un caractère qui n'est pas dans l'ensemble
%    \end{itemize}
  \end{itemize}
\end{frame}

% ---

\begin{frame}{L'astérisque (\texttt{*})}

  \begin{itemize}
    \setlength\itemsep{1em}

    \item L'astérisque \texttt{*} correspond à \textbf{n'importe quelle chaîne de caractères}, 
          y compris une chaîne vide (zéro ou plusieurs caractères)

    \pause

    \item Imaginons les fichiers suivants dans un dossier :
    \begin{terminal}
    \only<1,2,5,8>{
    report.txt report\_final.pdf image.jpg doc.txt
    }

    \only<3,4>{
    report{\color{red} .txt} report\_final.pdf image.jpg doc{\color{red} .txt}
    }

    \only<6,7>{
    {\color{red} rep}ort.txt {\color{red} rep}ort\_final.pdf image.jpg doc.txt
    }

    \only<9,10>{
    {\color{red}report.txt report\_final.pdf image.jpg doc.txt}
    }
    \end{terminal}

    \begin{terminal}
    \onslide<2->{
    \prompt\ \shcmd{ls} *.txt
    }

    \onslide<4->{
    report.txt doc.txt
    }
    \end{terminal}

    \begin{terminal}
    \onslide<5->{
    \prompt\ \shcmd{ls} rep*
    }

    \onslide<7->{
    report.txt report\_final.pdf
    }
    \end{terminal}

    \begin{terminal}
    \onslide<8->{
    \prompt\ \shcmd{ls} *
    }

    \onslide<10->{
    report.txt report\_final.pdf image.jpg doc.txt
    }
    \end{terminal}
  \end{itemize}
\end{frame}

% ---

\begin{frame}{Point d'interrogation (\texttt{?})}

  \begin{itemize}
    \setlength\itemsep{1em}

    \item Le point d'interrogation \texttt{?} correspond à \textbf{exactement un caractère}

    \pause

    \item Imaginons les fichiers :
    \begin{terminal}
    \only<1,2,5>{
    photo1.jpg photo2.jpg photoA.jpg photo10.jpg
    }

    \only<3,4>{
    {\color{red} photo}1{\color{red} .jpg} 
    {\color{red} photo}2{\color{red} .jpg}
    {\color{red} photo}A{\color{red} .jpg} 
    photo10.jpg
    }

    \only<6,7>{
    photo1.jpg photo2.jpg photoA.jpg {\color{red} photo}10{\color{red}.jpg}
    }
    \end{terminal}

    \begin{terminal}
    \onslide<2->{
    \prompt\ \shcmd{ls} photo?.jpg
    }

    \onslide<4->{
    photo1.jpg photo2.jpg photoA.jpg
    }
    \end{terminal}

    \begin{terminal}
    \onslide<5->{
    \prompt\ \shcmd{ls} photo??.jpg
    }

    \onslide<7->{
    photo10.jpg
    }
    \end{terminal}
  \end{itemize}

\end{frame}

% ---

\begin{frame}{Crochets (\texttt{[ ]})}

  \begin{itemize}
    \setlength\itemsep{1em}

    \item Les crochets \texttt{[...]} correspondent à \textbf{un seul caractère} parmi 
          ceux spécifiés dans l'ensemble

    \pause
  
    \item Imaginons les fichiers :
    \begin{terminal}
    \only<1,2,5,8>{
    rep1.pdf rep2.pdf rep3.pdf repA.pdf repB.pdf
    }

    \only<3,4>{
    {\color{red} rep}1{\color{red} .pdf}
    rep2.pdf
    {\color{red} rep}3{\color{red} .pdf}
    {\color{red} rep}A{\color{red} .pdf}
    repB.pdf
    }

    \only<6,7>{
    {\color{red} rep}1{\color{red} .pdf}
    {\color{red} rep}2{\color{red} .pdf}
    {\color{red} rep}3{\color{red} .pdf}
    repA.pdf
    repB.pdf
    }

    \only<9,10>{
    rep1.pdf
    rep2.pdf
    rep3.pdf
    {\color{red} rep}A{\color{red} .pdf}
    {\color{red} rep}B{\color{red} .pdf}
    }
    \end{terminal}

    \begin{terminal}
    \onslide<2->{
    \prompt\ \shcmd{ls} rep[13A].pdf
    }

    \onslide<4->{
    rep1.pdf rep3.pdf repA.pdf
    }
    \end{terminal}

    \begin{terminal}
    \onslide<5->{
    \prompt\ \shcmd{ls} rep[0-9].pdf
    }

    \onslide<7->{
    rep1.pdf rep2.pdf rep3.pdf
    }
    \end{terminal}

    \begin{terminal}
    \onslide<8->{
    \prompt\ \shcmd{ls} rep[A-Z].pdf
    }

    \onslide<10->{
    repA.pdf repB.pdf
    }
    \end{terminal}
  \end{itemize}

\end{frame}

% ---

\begin{frame}{Négation (\texttt{[!...]})}

  \begin{itemize}
    \setlength\itemsep{1em}

    \item le point d'exclamation ! (ou \texttt{\^{}}) placé en \textbf{première position} dans 
          les crochets correspond à \textbf{un seul caractère} qui n'est \textit{pas} 
          dans la liste ou l'intervalle spécifié
  
    \pause

    \item Imaginons les fichiers suivants :
    \begin{terminal}
    \only<1,2,5,8>{
    rapport1.log rapportA.log rapportB.log rapport\_final.log
    }

    \only<3,4>{
    rapport1.log {\color{red}rapport}A.log {\color{red}rapport}B.log {\color{red}rapport}\_final.log
    }

    \only<6,7>{
    {\color{red}r}apport1.log {\color{red}r}apportA.log {\color{red}r}apportB.log {\color{red}r}apport\_final.log
    }

    \only<9,10>{
    {\color{red}rapport}1.log rapportA.log rapportB.log {\color{red}rapport}\_final.log
    }
    \end{terminal}

    \begin{terminal}
    \onslide<2->{
    \prompt\ \shcmd{ls} rapport[!0-9].log
    }

    \onslide<4->{
    rapportA.log rapportB.log rapport\_final.log
    }
    \end{terminal}

    \begin{terminal}
    \onslide<5->{
    \prompt\ \shcmd{ls} [!abc]*
    }

    \onslide<7->{
    rapport1.log rapportA.log rapportB.log rapport\_final.log
    }
    \end{terminal}

    \begin{terminal}
    \onslide<8->{
    \prompt\ \shcmd{ls} rapport[!A-Z].log
    }

    \onslide<10->{
    rapport1.log rapport\_final.log
    }
    \end{terminal}
  \end{itemize}

\end{frame}

% ---

