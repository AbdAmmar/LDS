
\section{Introduction}

\begin{frame}{Brève histoire des ordinateurs}
\begin{itemize}
  \item Années 1940-50 : premiers ordinateurs
  \item Années 1960-70 : transistors, systèmes d’exploitation
  \item Années 1980 : ordinateurs personnels
  \item Années 2000 : Internet, cloud, automatisation
\end{itemize}
\end{frame}

\begin{frame}{TODO}
\begin{itemize}
  \item Parler directement avec le Hardware $\rightarrow$ n'est pas realiste ajd
  \item SATA: environ 400 pages pour a lire pour pouvoir sauvegarder un fichier sur le disque
  \item l'OS est une couche d'abstraction qui rend la vie simple pour le programmateur et l'utilisateur
  \item l'OS organize egalement qui utilise quoi
\end{itemize}
\end{frame}

\begin{frame}{TODO}
\begin{itemize}
  \item Windows, MAC OS, Linux
\end{itemize}
\end{frame}

\begin{frame}{Pq Linux}
\begin{itemize}
  \item Gratuit
  \item plusieurs utilisateurs
  \item l'importance d'open source
\end{itemize}
\end{frame}

\begin{frame}{Histoire d'UNIX (1970-1980)}
\begin{itemize}
  \item inspire par le projet MULTICS
  \item Fait en 1970 par Kenneth Thompson et Dennis Ritchie (Bell Labs)
  \item 2 Fork majeurs: System V (AT\&T) (eg, SOLARIS), BSD (Berkeley) (eg MAC OS)
  \item POSIX standard par IEEE
\end{itemize}
\end{frame}

\begin{frame}{Histoire de Linux (1980-aujourd'hui)}
\begin{itemize}
  \item Entre 1980-1991, c'etait devenu raisonnable d'avoir un ordi a la maison
  \item UNIX, DOS: payant
  \item Système libre et open-source
  \item GNU par Richard Stallman
  \item 1991: Linus Torvalds a develope le noyau (LINUX)
  \item Linux est ajd omnipresent.. quelques distributions
\end{itemize}
\end{frame}

