
% ---

\begin{frame}{Pourquoi a-t-on besoin d'un système d'exploitation ?}

  \centering
  \begin{minipage}{\textwidth}
    \begin{minipage}{0.37\textwidth}
      \centering
      \includegraphics[width=0.5\linewidth]{images/cns_zoom.png}

      %{\footnotesize\centering }
    \end{minipage}
    \hfill
    \begin{minipage}{0.62\textwidth}
      \centering
      \includegraphics[width=\linewidth]{images/Fig1.6_Tanenbaum_Bos.png}

      %{\footnotesize\centering }
    \end{minipage}
  \end{minipage}

  \vspace{0.25cm}

  \begin{itemize}
    \item Exemple : calculer $x + y$
    \begin{enumerate}
      {\footnotesize
      \pause
      \item les données et le programme sont saisis via le clavier
      \pause
      \item le programme et les données sont chargés en mémoire
      \pause
      \item le processeur (CPU) exécute les instructions
      \pause
      \item le résultat est affiché à l'écran
      }
    \end{enumerate}

    %\vspace{0.1cm}
    %\item Rôle du système d'exploitation :
    %\begin{itemize}
    %  {\footnotesize
    %  \item gérer le clavier, l'écran et les autres périphériques
    %  \item gérer la mémoire et le processeur
    %  \item servir d'intermédiaire entre le matériel et les programmes
    %  }
    %\end{itemize}
  \end{itemize}
\end{frame}

% ---

\begin{frame}{Exemple : abstraction d'un disque dur}
  \begin{itemize}
    \item Les disques durs SATA (Serial ATA) sont utilisés sur la plupart des ordinateurs
    \begin{itemize}
      {\footnotesize
      \item première couche d'abstraction : disk driver
      \item deuxième couche d'abstraction : fichiers
      }
    \end{itemize}
  \end{itemize}
  
  \vspace{0.1cm}
  \centering
  \begin{minipage}{\textwidth}
    \begin{minipage}{0.45\textwidth}
      \centering
      \includegraphics[width=0.7\linewidth]{images/ST3400820AS.jpg}

      %{\footnotesize\centering }
    \end{minipage}
    \hfill
    \begin{minipage}{0.45\textwidth}
      \centering
      \includegraphics[width=0.8\linewidth]{images/SATA_Anderson.jpg}

      {\footnotesize\centering Anderson, 2007 (environ 450 pages)}
    \end{minipage}
  \end{minipage}
\end{frame}

% ---

\begin{frame}{Le système d'exploitation}
  \begin{itemize}
    \item Le rôle du système d'exploitation est de :
    \begin{itemize}
      {\footnotesize
        \item fournir des abstractions simples (par exemple : fichiers, processus)
        %\item implémenter et gérer ces abstractions
        \item assurer la communication entre les applications et le matériel
        %\item s'appuyer sur des pilotes pour contrôler les périphériques
      }
    \end{itemize}
  \end{itemize}
  
  \vspace{0.1cm}
  \centering
  \begin{minipage}{0.99\textwidth}
    \centering
    \includegraphics[width=0.5\linewidth]{images/Fig1.1_Tanenbaum_Bos.png}

    %{\footnotesize\centering }
  \end{minipage}
\end{frame}

% ---

\begin{frame}{Histoire d'\texttt{UNIX} (1970--1980)}
  \begin{itemize}
    \item Inspiré par le projet \texttt{MULTICS}
    \item Développé à partir de 1969 par Kenneth Thompson et Dennis Ritchie (Bell Labs)
    \item Initialement diffusé avec le code source dans les universités, puis commercialisé par AT\&T
  \end{itemize}

  \vspace{0.1cm}
  \centering
  \begin{minipage}{0.99\textwidth}
    \centering
    \includegraphics[width=0.5\linewidth]{images/Fig1.30_Fox.png}

    %{\footnotesize\centering }
  \end{minipage}
\end{frame}

% ---

\begin{frame}{Histoire d'\texttt{UNIX} (1970--1980)}
  \begin{itemize}
    \item Deux grandes familles (forks) de systèmes de type \texttt{UNIX} :
    \begin{itemize}
      {\footnotesize
      \item \texttt{System V} (AT\&T) - ex. : \texttt{Solaris}
      \item \texttt{BSD} (Université de Berkeley) - à l'origine de \texttt{FreeBSD}, \texttt{macOS}, etc.
      }
    \end{itemize}

    \pause
    \item POSIX : ensemble de standards définis par l'IEEE pour assurer la portabilité des programmes

    \pause
    \item Exemple POSIX : tout processus possède les descripteurs :
    \begin{itemize}
      {\footnotesize
      \item \texttt{stdin} (0) : entrée standard (clavier)
      \item \texttt{stdout} (1) : sortie standard (écran)
      \item \texttt{stderr} (2) : sortie d'erreur
      }
    \end{itemize}
  \end{itemize}
\end{frame}

% ---

\begin{frame}{Histoire de Linux (années 1980--aujourd'hui)}
  \begin{itemize}
    \item Années 1980--1990 : les ordinateurs personnels deviennent accessibles au grand public

    \pause
    \item Les systèmes d'exploitation existants sont majoritairement propriétaires :
    \begin{itemize}
      {\footnotesize
      \item \texttt{UNIX} : cher, réservé aux entreprises et universités
      \item \texttt{DOS} : propriétaire (Microsoft)
      }
    \end{itemize}

    \pause
    \item Système d'exploitation open source (\textbf{libre}, et pas juste gratuit)
    \begin{itemize}
      {\footnotesize
      \item transparence et contrôle du code
      \item sécurité renforcée (audit public)
      \item coût réduit (logiciel libre)
      \item flexibilité et personnalisation
      \item indépendance technologique
      \item résistance aux monopoles
      }
    \end{itemize}
  \end{itemize}
\end{frame}

% ---

\begin{frame}{Histoire de Linux (années 1980--aujourd'hui)}
  \begin{itemize}
    \item Années 1980 : \texttt{GNU} (GNU's Not Unix) lancé par Richard Stallman
    \begin{itemize}
      {\footnotesize
      \item compilateur : gcc et bibliothèque standard : glibc
      \item shell : bash et les outils : ls, cp, grep, make, etc.
      \item Éditeur : emacs
      \item MAIS, pas de noyau !
      }
    \end{itemize}
  \end{itemize}

  \vspace{0.1cm}
  \centering
  \begin{minipage}{\textwidth}
    \begin{minipage}{0.45\textwidth}
      \centering
      \includegraphics[width=0.6\linewidth]{images/700px-The_GNU_logo.png}

      {\footnotesize\centering symbole de \texttt{GNU} (gnous, espèces de bovidés)}
    \end{minipage}
    \hfill
    \begin{minipage}{0.45\textwidth}
      \centering
      \includegraphics[width=0.6\linewidth]{images/Richard_Stallman_at_LibrePlanet_2019.jpg}

      {\footnotesize\centering Richard Stallman (1985--2019)}
    \end{minipage}
  \end{minipage}
\end{frame}

% ---

\begin{frame}{Histoire de Linux (années 1980--aujourd'hui)}
  \begin{itemize}
    \item 1991 : Linus Torvalds développe le noyau \texttt{Linux}
    \begin{itemize}
      {\footnotesize
      \item des dizaines de millions de lignes de code
      \item principalement écrit en langage \texttt{C}
      }
    \end{itemize}

    \item L'association de \texttt{GNU} et \texttt{Linux} donne naissance à un OS libre complet
  \end{itemize}

  \vspace{0.1cm}
  \centering
  \begin{minipage}{\textwidth}
    \begin{minipage}{0.45\textwidth}
      \centering
      \includegraphics[width=0.6\linewidth]{images/Tux.png}

      {\footnotesize\centering symbole de \texttt{Linux}}
    \end{minipage}
    \hfill
    \begin{minipage}{0.45\textwidth}
      \centering
      \includegraphics[width=0.6\linewidth]{images/LinuxCon_Europe_Linus_Torvalds.jpg}

      {\footnotesize\centering Linus Torvalds}
    \end{minipage}
  \end{minipage}
\end{frame}

% ---

\begin{frame}{Histoire de Linux (années 1980--aujourd'hui)}
  \begin{itemize}
    \item Linux est aujourd'hui omniprésent :
    \begin{itemize}
      {\footnotesize
      \item serveurs et cloud
      \item supercalculateurs
      \item smartphones (Android)
      \item systèmes embarqués
      }
    \end{itemize}

    \item Il existe de nombreuses distributions :
    \begin{itemize}
      \footnotesize
      \item \texttt{Ubuntu}, \texttt{Debian}, \texttt{Fedora}, \texttt{Arch}, \texttt{Kali}, etc.
    \end{itemize}
  \end{itemize}

  \vspace{0.1cm}
  \centering
  \begin{minipage}{\textwidth}
    \begin{minipage}{0.15\textwidth}
      \centering
      \includegraphics[width=0.8\linewidth]{images/ubuntu_logo.png}

      {\footnotesize\centering \texttt{Ubuntu}}
    \end{minipage}
    \hfill
    \begin{minipage}{0.15\textwidth}
      \centering
      \includegraphics[width=0.8\linewidth]{images/debian_logo.png}

      {\footnotesize\centering \texttt{Ubuntu}}
    \end{minipage}
    \hfill
    \begin{minipage}{0.15\textwidth}
      \centering
      \includegraphics[width=0.8\linewidth]{images/fedora_logo.png}

      {\footnotesize\centering \texttt{Ubuntu}}
    \end{minipage}
    \hfill
    \begin{minipage}{0.15\textwidth}
      \centering
      \includegraphics[width=0.8\linewidth]{images/arch_logo.jpeg}

      {\footnotesize\centering \texttt{Ubuntu}}
    \end{minipage}
    \hfill
    \begin{minipage}{0.15\textwidth}
      \centering
      \includegraphics[width=0.8\linewidth]{images/kali_logo.png}

      {\footnotesize\centering \texttt{Ubuntu}}
    \end{minipage}
    \hfill
  \end{minipage}
\end{frame}

% ---

