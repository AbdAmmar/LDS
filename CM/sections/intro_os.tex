
% ---

\begin{frame}{l'OS}
\begin{itemize}
  \item 1+1 = ?
  \begin{itemize}
    \item inputs du clavier
    \item ecrire les instruction en lagages machines (assembly)
    \item envoyer le code et les donnes en format binaire via le bus a la memoire
    \item envoyer les instructions vers le CPU
    \item recuperer le resultat et l'afficher
  \end{itemize}

  {
  \centering
  \begin{minipage}{0.37\textwidth}
    \centering
    \includegraphics[width=0.5\linewidth]{images/cns_zoom.png}
    %{\footnotesize\centering }
  \end{minipage}
  %\hfill
  \begin{minipage}{0.52\textwidth}
    \centering
    \includegraphics[width=\linewidth]{images/Fig1.6_Tanenbaum_Bos.png}
    %{\footnotesize\centering }
  \end{minipage}
  }
\end{itemize}
\end{frame}

% ---

\begin{frame}{l'OS}
\begin{itemize}
  \item SATA (Serial ATA) hard disks used on most computers
  \begin{itemize}
    {\footnotesize
    \item premier couche d'abstraction: disk driver
    \item deuxiem couche d'abstraction: files
    }
  \end{itemize}

  {
  \centering
  \begin{minipage}{0.45\textwidth}
    \centering
    \includegraphics[width=0.8\linewidth]{images/ST3400820AS.jpg}
    %{\footnotesize\centering }
  \end{minipage}
  %\hfill
  \begin{minipage}{0.45\textwidth}
    \centering
    \includegraphics[width=0.9\linewidth]{images/SATA_Anderson.jpg}
    {\footnotesize\centering Anderson 2007 (environ 450 pages)}
  \end{minipage}
  }
\end{itemize}
\end{frame}

% ---

\begin{frame}{l'OS}
\begin{itemize}
  \item Le rôle d'OS est de
  \begin{itemize}
    {\footnotesize
      \item créer de bonnes abstractions (par exemple, fichier)
      \item puis d'implémenter et de gérer les objets abstraits ainsi créés (pour parler avec les pilotes et le materiel)
    }
  \end{itemize}

  {
  \centering
  \begin{minipage}{0.99\textwidth}
    \centering
    \includegraphics[width=0.5\linewidth]{images/Fig1.1_Tanenbaum_Bos.png}
    %{\footnotesize\centering }
  \end{minipage}
  }
\end{itemize}
\end{frame}

% ---

\begin{frame}{Histoire d'UNIX (1970-1980)}
\begin{itemize}
  \item inspire par le projet MULTICS
  \item Fait en 1970 par Kenneth Thompson et Dennis Ritchie (Bell Labs)
  \item 2 Fork majeurs: System V (AT\&T) (eg, SOLARIS), BSD (Berkeley) (eg MAC OS)
  \item POSIX standard par IEEE
\end{itemize}
\end{frame}

% ---

\begin{frame}{Histoire de Linux (1980-aujourd'hui)}
\begin{itemize}
  \item Entre 1980-1991, c'etait devenu raisonnable d'avoir un ordi a la maison
  \item UNIX, DOS: payant
  \item Système libre et open-source
  \item GNU par Richard Stallman
  \item 1991: Linus Torvalds a develope le noyau (LINUX)
  \item Linux est ajd omnipresent.. quelques distributions
\end{itemize}
\end{frame}

% ---

\begin{frame}{Avantages d'un OS open source}
\begin{itemize}
  \item Transparence et contrôle
  \item Sécurité renforcée
  \item Coût réduit
  \item Flexibilité et personnalisation
  \item Indépendance technologique (pas de dépendance géopolitique)
  \item Résistance aux monopoles
\end{itemize}
\end{frame}

% ---

