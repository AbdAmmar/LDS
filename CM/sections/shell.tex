
% ---

\begin{frame}{Le shell}
\begin{itemize}
  \item Le shell est l'interface principal avec l'OS
  \item terminal: program pour interagir avec le shell
  \item GUI (GNOME, KDE) est un autre program qui tourne au-dessus de l'OS (ou pas)

  {
  \centering
  \begin{minipage}{0.45\textwidth}
    \centering
    \includegraphics[width=0.99\linewidth]{images/Fig2_Barrett.png}
    %{\footnotesize\centering }
  \end{minipage}
  \hfill
  \begin{minipage}{0.45\textwidth}
    \centering
    \includegraphics[width=0.99\linewidth]{images/terminal.png}
    %{\footnotesize\centering }
  \end{minipage}
  }
\end{itemize}
\end{frame}

% ---

\begin{frame}{TODO}
\begin{itemize}
  \item sh: original Unix shell program written by Steve Bourne
  \item bash (Bourne Again SHell): program de GNU
  \item csh, ksh, zsh (macOS)

  {
  \centering
  \begin{minipage}{\textwidth}
    \centering
    \includegraphics[width=0.49\linewidth]{images/terminal_bash.png}
    %{\footnotesize\centering }
  \end{minipage}
  %\hfill
  }
\end{itemize}
\end{frame}

% ---

%\begin{frame}{TODO}
%  \begin{itemize}
%    \item Commande \textit{built-in} (ou \textit{keyword}) vs commande externe
%    \item An executable program (in /usr/bin): binaires ou scripts (ls, cp, which)
%    \item A command built into the shell itself (type, cd, help)
%    \item A shell function
%    \item An alias
%  \end{itemize}
%\end{frame}

% ---

