
% ---

\begin{frame}{Le shell}
  \begin{itemize}
    \item Le shell est l'interface fondamentale du système d'exploitation
    \item Le terminal est un programme permettant d'interagir avec le shell (en ligne de commande)
    \item Les interfaces graphiques (GUI), comme \texttt{GNOME} ou \texttt{KDE}, sont optionnelles
  \end{itemize}

  \vspace{0.10 cm}
  \centering
  \begin{minipage}{0.45\textwidth}
    \centering
    \includegraphics[width=0.9\linewidth]{images/Fig2_Barrett.png}

    %{\footnotesize\centering }
  \end{minipage}
  \hfill
  \begin{minipage}{0.45\textwidth}
    \centering
    \includegraphics[width=0.9\linewidth]{images/terminal.png}

    %{\footnotesize\centering }
  \end{minipage}
\end{frame}

% ---

\begin{frame}{Quelques shells UNIX}

  \begin{itemize}
    \item \texttt{sh} : Bourne shell, shell UNIX historique (Steve Bourne)

    \pause
    \item \texttt{bash} (\textit{Bourne Again SHell}) :
    \begin{itemize}
      \footnotesize
      \item shell du projet \texttt{GNU}
      \item très répandu sur les systèmes Linux
    \end{itemize}

    \pause
    \item Autres shells :
    \begin{itemize}
      \footnotesize
      \item \texttt{csh} (C shell)
      \item \texttt{ksh} (Korn shell)
      \item \texttt{zsh} (shell par défaut sur macOS)
    \end{itemize}
  \end{itemize}

  \vspace{0.10 cm}
  \centering
  \begin{minipage}{\textwidth}
    \centering
    \includegraphics[width=0.49\linewidth]{images/terminal_bash.png}

    %{\footnotesize\centering }
  \end{minipage}
\end{frame}

%\begin{frame}{TODO}
%  \begin{itemize}
%    \item Commande \textit{built-in} (ou \textit{keyword}) vs commande externe
%    \item An executable program (in /usr/bin): binaires ou scripts (ls, cp, which)
%    \item A command built into the shell itself (type, cd, help)
%    \item A shell function
%    \item An alias
%  \end{itemize}
%\end{frame}

% ---

\begin{frame}{Premières commandes du shell}
  \begin{itemize}
    \item \texttt{pwd} : affiche le répertoire courant
    \item \texttt{ls} : liste le contenu d'un répertoire
    \item \texttt{cd} : change de répertoire
    \item \texttt{mkdir} : crée un répertoire
    \item \texttt{touch} : crée un fichier vide
  \end{itemize}
\end{frame}

% ---

%\begin{frame}{Arborescence et chemins}
%  \begin{itemize}
%    \item Le système de fichiers est organisé en arborescence
%    \item La racine est notée \texttt{/}
%    \item Deux types de chemins :
%    \begin{itemize}
%      \footnotesize
%      \item Chemin absolu : commence par \texttt{/}
%      \item Chemin relatif : dépend du répertoire courant
%    \end{itemize}
%    \item Raccourcis :
%    \begin{itemize}
%      \footnotesize
%      \item \texttt{.} : répertoire courant
%      \item \texttt{..} : répertoire parent
%    \end{itemize}
%  \end{itemize}
%\end{frame}

% ---

%\begin{frame}{Fichiers et permissions}
%  \begin{itemize}
%    \item Chaque fichier possède des permissions :
%    \begin{itemize}
%      \footnotesize
%      \item \texttt{r} : lecture
%      \item \texttt{w} : écriture
%      \item \texttt{x} : exécution
%    \end{itemize}
%    \item Trois niveaux :
%    \begin{itemize}
%      \footnotesize
%      \item utilisateur (u), groupe (g), autres (o)
%    \end{itemize}
%    \item Commandes utiles :
%    \begin{itemize}
%      \footnotesize
%      \item \texttt{ls -l}
%      \item \texttt{chmod}
%    \end{itemize}
%  \end{itemize}
%\end{frame}

% ---

%\begin{frame}{Redirections et pipes}
%
%\begin{itemize}
%  \item Entrée / sortie standard :
%  \begin{itemize}
%    \footnotesize
%    \item \texttt{stdin}, \texttt{stdout}, \texttt{stderr}
%  \end{itemize}
%  \item Redirections :
%  \begin{itemize}
%    \footnotesize
%    \item \texttt{>} : redirige la sortie
%    \item \texttt{<} : redirige l'entrée
%  \end{itemize}
%  \item Pipe :
%  \begin{itemize}
%    \footnotesize
%    \item \texttt{|} : relie la sortie d'un programme à l'entrée d'un autre
%  \end{itemize}
%\end{itemize}
%
%\end{frame}

% ---

%\begin{frame}{Processus et contrôle}
%  \begin{itemize}
%    \item Une commande lancée = un processus
%    \item Commandes utiles :
%    \begin{itemize}
%      \footnotesize
%      \item \texttt{ps} : liste des processus
%      \item \texttt{top} : processus en temps réel
%      \item \texttt{kill} : termine un processus
%    \end{itemize}
%    \item Exécution en arrière-plan :
%    \begin{itemize}
%      \footnotesize
%      \item \texttt{\&}
%    \end{itemize}
%  \end{itemize}
%\end{frame}

% ---

