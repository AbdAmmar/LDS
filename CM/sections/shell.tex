
\section{Le Shell}


\begin{frame}{TODO}
\begin{itemize}
  \item le shell ne fait pas partir du noyau de l'OS
  \item le shell est l'interface principal avec l'OS
  \item le shell possede un terminal pour le standard I/O
  \item sh: original Unix shell program written by Steve Bourne
  \item bash (Bourne Again SHell): program de GNU
  \item csh, ksh, zsh (macOS)
\end{itemize}
\end{frame}

\begin{frame}{TODO}
\begin{itemize}
  \item terminal: program pour interagir avec le shell
  \item GUI est un autre program qui tourne au-dessus de l'OS (ou pas)
  \item En Linux, 2 choix de GUI: GNOME, KDE, ou rien (le hacker n'utilise jamais une souris)
\end{itemize}
\end{frame}

\begin{frame}{TODO}
  \begin{itemize}
    \item Commande \textit{built-in} (ou \textit{keyword}) vs commande externe
    \item An executable program (in /usr/bin): binaires ou scripts (ls, cp, which)
    \item A command built into the shell itself (type, cd, help)
    \item A shell function
    \item An alias
  \end{itemize}
\end{frame}

\begin{frame}{TODO: commande principale}
  \begin{itemize}
    \item
  \end{itemize}
  pour avoir plus d'info sur les commandes \textit{built-in}: help
  Sinon, man ou info
\end{frame}

