
\begin{frame}{Evaluation d'ordinateurs}
\begin{itemize}
  \item Une machine à différences par Charles Babbage
  \begin{itemize}
    {\footnotesize
    \item présentée en 1822
    \item fonctionnemeent mécanique (Blaise Pascal)
    \item calcul des fonctions polynomiales
    \item commercialisée entre 1855-1930
    }
  \end{itemize}

  \centering
  \includegraphics[width=0.40\textwidth]{images/Fig1.13_Fox.png}
\end{itemize}
\end{frame}

% ---

\begin{frame}{Evaluation d'ordinateurs}
\begin{itemize}
  \item La machine analytique par Charles Babbage
  \begin{itemize}
    {\footnotesize
    \item fonctionnemeent mécanique
    \item Harvard architecture (données et programme stockés séparément)
    \item Ada Lovelace travaillais sur la partie "software"
    \item projet jamais termine
    }
  \end{itemize}

  \centering
  \includegraphics[width=0.35\textwidth]{images/Fig1.14_Fox.png}

  {\footnotesize\centering Une reconstruction partielle moderne de la machine analytique de Babbage}
\end{itemize}
\end{frame}

% ---

\begin{frame}{Evaluation d'ordinateurs}
\begin{itemize}
  \item L'ère électrique (\texttt{Colossus}-1943, Tommy Flowers, Max Newman, Alan Turing, etc.)
  \begin{itemize}
    {\footnotesize
    \item tube a vide inventé par John Fleming en 1904
    \item fils chaufante: passe ou pas un courant
    \item transition de machines électromécaniques vers des machines purement electronique
    \item utilisation cryptographi en deuxieme guerre mondiale
    }
  \end{itemize}

  \centering
  \includegraphics[width=0.40\textwidth]{images/Fig1.22_Fox.png}

  {\footnotesize\centering \texttt{Colossus}, Bletchley Park, 1943 avec operators Dorothy Du Boisson and Elsie Booker}
\end{itemize}
\end{frame}

% ---

\begin{frame}{Evaluation d'ordinateurs}
\begin{itemize}
  \item ENIAC (Electronic Numerical Integrator and Computer), 1945
  \begin{itemize}
    {\footnotesize
    \item utilisation principalement militaire (calculs balistiques, bombe hydrogenoide)
    \item machine à tubes à vide américaine
    %\item developpe par John Mauchly et J. Presper Eckert
    \item conception basée sur la machine analytique de Babbage (Harvard architecture)
    }
  \end{itemize}

  \centering
  \includegraphics[width=0.35\textwidth]{images/Fig1.23_Fox.png}

  {\footnotesize\centering ENIAC avec les informaticiennes Betty Jean Jennings and Frances Bilas}
\end{itemize}
\end{frame}

% ---

\begin{frame}{Evaluation d'ordinateurs}
\begin{itemize}
  \item The Baby ENIAC (Electronic Numerical Integrator and Computer), 1945
  \begin{itemize}
    {\footnotesize
    \item machine à tubes à vide anglaise
    \item machine 32-bit 
    \item Architecture Von Neumann (données et programme stockés ensemble)
    }
  \end{itemize}

  \centering
  \includegraphics[width=0.35\textwidth]{images/Fig1.24_Fox.png}

  {\footnotesize\centering Reconstitution de "The baby" au Musée des sciences et de l'industrie de Manchester}
\end{itemize}
\end{frame}

% ---

\begin{frame}{Evaluation d'ordinateurs}
\begin{itemize}
  \item L'ère du transistor (1960): meme fonction que les tubes a vide, mais
  \begin{itemize}
    {\footnotesize
    \item plus petit
    \item plus rapide
    \item moins chere
    \item consomme moins d'energie
    \item plus fiable
    }
  \end{itemize}

  \centering
  \begin{minipage}{0.48\textwidth}
    \centering
    \includegraphics[width=\linewidth]{images/Fig1.27_Fox.png}
    {\footnotesize\centering Un mini-ordinateur PDP-11 des années 1960 à transistors}
  \end{minipage}\hfill
  \begin{minipage}{0.48\textwidth}
    \centering
    \includegraphics[width=\linewidth]{images/Fig1.28_Fox.png}
    {\footnotesize\centering Margaret Hamilton avec une impression de son programme d'assemblage complet pour Apollo 11}
  \end{minipage}
\end{itemize}
\end{frame}

% ---

\begin{frame}{Evaluation d'ordinateurs}
\begin{itemize}
  \item La technologie des circuits intégrés (1970) permet de miniaturiser les circuits électriques à base de transistors

  \centering
  \includegraphics[width=0.30\textwidth]{images/Fig1.31_Fox.png}

  {\footnotesize\centering L'informatique domestique dans les années 1980}
\end{itemize}
\end{frame}

% ---

%\begin{frame}{Evaluation d'ordinateurs}
%\begin{itemize}
%  \item MONIAC (ordinateur hydrolique) par Bill Phillips (1949) \\[0.2 cm] 
%
%  \centering
%  \includegraphics[width=0.25\textwidth]{images/Fig1.1_Fox.png}
%
%  \medskip
%  {\footnotesize\centering MONIAC\footnote{Monetary National Income Analogue Computer} avec Bill Phillips (1949)\par}
%\end{itemize}
%\end{frame}

% ---

%\begin{frame}{Brève histoire des ordinateurs}
%\begin{itemize}
%  \item Années 1940-50 : premiers ordinateurs
%  \item Années 1960-70 : transistors, systèmes d’exploitation
%  \item Années 1980 : ordinateurs personnels
%  \item Années 2000 : Internet, cloud, automatisation
%\end{itemize}
%\end{frame}

