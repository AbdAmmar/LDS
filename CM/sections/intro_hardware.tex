
% ---

\begin{frame}{Ordinateur mécanique}
  \begin{itemize}
    \item \texttt{La machine à différences} (Charles Babbage, 1822)
    \begin{itemize}
      {\footnotesize
      \item fonctionnement mécanique (Blaise Pascal)
      \item calcul des fonctions polynomiales
      \item largement utilisée entre 1855 et 1930
      }
    \end{itemize}
  \end{itemize}
  
  \vspace{0.25 cm}
  \centering
  \begin{minipage}{\textwidth}
    \begin{minipage}{0.45\textwidth}
      \centering
      \includegraphics[width=0.99\textwidth]{images/Fig1.13_Fox.png}

      {\footnotesize\centering \texttt{Machine à différences}}
    \end{minipage}
    \hfill
    \begin{minipage}{0.45\textwidth}
      \centering
      \includegraphics[width=0.70\textwidth]{images/Charles_Babbage_1860.jpg}

      {\footnotesize\centering Charles Babbage (1791--1871)}
    \end{minipage}
  \end{minipage}
\end{frame}

% ---

\begin{frame}{Ordinateur mécanique}
  \begin{itemize}
    \item \texttt{La machine analytique} (Charles Babbage)
    \begin{itemize}
      {\footnotesize
      \item fonctionnemeent mécanique
      \item Architecture de type Harvard : données et programme stockés séparément
      \item Ada Lovelace pour la partie "software" (premier programme informatique)
      \item Projet jamais terminé, mais concept révolutionnaire
      }
    \end{itemize}
  \end{itemize}
  
  \vspace{0.15 cm}
  \centering
  \begin{minipage}{\textwidth}
    \begin{minipage}{0.45\textwidth}
      \centering
      \includegraphics[width=0.95\textwidth]{images/Fig1.14_Fox.png}

      {\footnotesize\centering Reconstruction partielle moderne de \texttt{la machine analytique}}
    \end{minipage}
    \hfill
    \begin{minipage}{0.45\textwidth}
      \centering
      \includegraphics[width=0.70\textwidth]{images/Ada_Byron_daguerreotype_by_Antoine_Claudet_1843_or_1850.png}

      {\footnotesize\centering Ada Lovelace (1815--1852)}
    \end{minipage}
  \end{minipage}
\end{frame}

% ---

\begin{frame}{L'ère de l'électronique}
  \begin{itemize}
    \item Tubes à vide
    \begin{itemize}
      {\footnotesize
      \item inventés par John Fleming en 1904
      \item un fil chauffant laisse passer ou non un courant (0 ou 1)
      }
    \end{itemize}
    \item \texttt{Colossus}-1943 : Tommy Flowers, Max Newman, Alan Turing, etc.
    \begin{itemize}
      {\footnotesize
      \item machines électromécaniques $\rightarrow$ machines purement électroniques
      \item utilisation en cryptographie pendant la Seconde Guerre mondiale
      }
    \end{itemize}
  \end{itemize}
  
  \vspace{0.15 cm}
  \centering
  \begin{minipage}{\textwidth}
    \centering
    \includegraphics[width=0.40\textwidth]{images/Fig1.22_Fox.png}

    {\footnotesize\centering \texttt{Colossus} à Bletchley Park (1943), avec les opératrices Dorothy Du Boisson et Elsie Booker}
  \end{minipage}
\end{frame}

% ---

\begin{frame}{L'ère de l'électronique}
  \begin{itemize}
    \item \texttt{ENIAC} (Electronic Numerical Integrator and Computer), 1945
    \begin{itemize}
      {\footnotesize
      \item machine américaine à tubes à vide
      \item premier ordinateur électronique numérique généraliste
      \item utilisation principalement militaire (calculs balistiques, simulations bombe à hydrogène)
      \item développé par John Mauchly et J. Presper Eckert
      \item architecture inspirée de la machine analytique de Babbage %\textbf{architecture Harvard}
      }
    \end{itemize}
  \end{itemize}
  

  \vspace{0.15 cm}
  \centering
  \begin{minipage}{\textwidth}
    \centering
    \includegraphics[width=0.35\textwidth]{images/Fig1.23_Fox.png}

    {\footnotesize\centering ENIAC avec les programmatrices Betty Jean Jennings and Frances Bilas}
  \end{minipage}
\end{frame}

% ---

\begin{frame}{L'ère de l'électronique}
  \begin{itemize}
    \item \texttt{Manchester Baby} (Small-Scale Experimental Machine, 1948)
    \begin{itemize}
      {\footnotesize
      \item machine anglaise à tubes à vide
      \item architecture Von Neumann (données et programme stockés ensemble)
      }
    \end{itemize}
  \end{itemize}
  
  \vspace{0.15 cm}
  \centering
  \begin{minipage}{\textwidth}
    \centering
    \includegraphics[width=0.35\textwidth]{images/Fig1.24_Fox.png}

    {\footnotesize Reconstitution du \texttt{Manchester Baby} au Musée des sciences et de l'industrie de Manchester}
  \end{minipage}
\end{frame}

% ---

\begin{frame}{L'ère du transistor (années 1960)}
  \begin{itemize}
    \item Les transistors remplacent les tubes à vide :
    \begin{itemize}
      {\footnotesize
      \item plus petits et plus légers
      \item plus rapides et plus efficaces
      \item moins chers à produire
      \item consommation d'énergie réduite
      \item fiabilité accrue
      }
    \end{itemize}
  \end{itemize}
  
  \vspace{0.05 cm}
  \centering
  \begin{minipage}{0.48\textwidth}
    \centering
    \includegraphics[width=0.80\linewidth]{images/Fig1.27_Fox.png}

    {\footnotesize Mini-ordinateur PDP-11 (années 1960)}
  \end{minipage}
  \hfill
  \begin{minipage}{0.48\textwidth}
    \centering
    \includegraphics[width=0.60\linewidth]{images/Fig1.28_Fox.png}

    {\footnotesize Margaret Hamilton avec le code source du logiciel de bord d'Apollo 11 (1969)}
  \end{minipage}
\end{frame}

% ---

\begin{frame}{La révolution de la miniaturisation}
\begin{itemize}
  \item Les circuits intégrés (1970) permettent de :
  \begin{itemize}
    \item miniaturiser les circuits électroniques à base de transistors
    \item intégrer des milliers (puis des millions) de composants sur une seule puce
    \item réduire les coûts, la consommation d'énergie et la taille des ordinateurs
  \end{itemize}

  \vspace{0.15 cm}
  \centering
  \begin{minipage}{\textwidth}
    \centering
    \includegraphics[width=0.25\textwidth]{images/Fig1.31_Fox.png}

    {\footnotesize\centering L'informatique domestique dans les années 1980}
  \end{minipage}
\end{itemize}
\end{frame}

% ---

